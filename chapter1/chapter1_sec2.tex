Next, we approach teh theory of decision making from choice behavior itself. Formally, choice behavior is represented by means of a \textit{choice structure} $(\mathcal{B},C(\cdot))$. Now, we define choice structure $(\mathcal{B},C(\cdot))$:
\begin{definition}
    A choice structure $(\mathcal{B},C(\cdot))$ has two ingredients:
    \begin{enumerate}
        \item[-] $\mathcal{B}\subset \mathcal{P}(X)\setminus\varnothing$, where $\mathcal{P}(X)$ is the power set of $X$. This means, every element $B\in \mathcal{B}$ is a subset of $X$\footnote{The elements $B\in\mathcal{B}$ are so-called \textit{budget sets}. The budget sets in $\mathcal{B}$ should be thought of as an exhaustive listing of all the choice experiments that can be achieved, but it is possible that some subsets of $X$ are not achievable.}.
        \item[-] $C(\cdot)$ is a \textit{choice rule correspondence} that assigns a nonempty set of chosen elements $C(B)\subset B$, $\forall B\in \mathcal{B}$\footnote{The choice set $C(B)$ can contain a single element, which is the choice among the alternatives in $B$. BUT, $C(B)$ can contain multiple elements, then elements of $C(B)$ are the \textit{acceptable alternatives} in $B$.}.
    \end{enumerate}
\end{definition}