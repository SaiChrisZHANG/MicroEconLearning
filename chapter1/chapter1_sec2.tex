Next, we approach the theory of decision making from choice behavior itself. Formally, choice behavior is represented by means of a \textit{choice structure} $(\mathcal{B},C(\cdot))$. Now, we define choice structure $(\mathcal{B},C(\cdot))$:
\begin{definition}
    A choice structure $(\mathcal{B},C(\cdot))$ has two ingredients:
    \begin{enumerate}
        \item[-] $\mathcal{B}\subset \mathcal{P}(X)\setminus\varnothing$, where $\mathcal{P}(X)$ is the power set of $X$. This means, every element $B\in \mathcal{B}$ is a subset of $X$\footnote{The elements $B\in\mathcal{B}$ are so-called \textit{budget sets}. The budget sets in $\mathcal{B}$ should be thought of as an exhaustive listing of all the choice experiments that can be achieved, but it is possible that some subsets of $X$ are not achievable.}.
        \item[-] $C(\cdot)$ is a \textit{choice rule correspondence} that assigns a nonempty set of chosen elements $C(B)\subset B$, $\forall B\in \mathcal{B}$\footnote{The choice set $C(B)$ can contain a single element, which is the choice among the alternatives in $B$. BUT, $C(B)$ can contain multiple elements, then elements of $C(B)$ are the \textit{acceptable alternatives} in $B$.}.
    \end{enumerate}
\end{definition}

Now we discuss the CORE assumption in this section: the Weak Axiom of Revealed Preference (WARP):
\begin{definition}
    A choice set $(\mathcal{B},C(\cdot))$ satisfies WARP if:
    \begin{enumerate}
        \item[-] $\forall B,B^{\prime}$ and $x,y\in B\cap B^{\prime}$, $x\in C(B),y\in C(B^{\prime})\Rightarrow x\in C(B^{\prime})$
    \end{enumerate}
\end{definition}

Or in words, WARP requires that if $x$ is chosen from some alternatives where $y$ is also available, then there can be NO budget set containing both $x$ and $y$
but only $y$ is chosen.

Following WARP, define the \textit{reveal preference relation} $\succsim^*$ as:
\begin{definition}
    Given a choice structure $(\mathcal{B},C(\cdot))$, $x\succsim^*y\Leftrightarrow \exists B\in\mathcal{B}$ s.t. $x,y\in B\land x\in C(B)$
\end{definition}
In words, $x$ is revealed at least as good as $y$.

With revealed preference defiend, we can rephrase WARP as: \textit{If x is revealed at least as good as y, then y \textbf{cannot} be revealed preferred to x}. Hence, $\succsim^*$ is not symmetric.

One thing to remember is that $\succsim^*$ need not be either complete or transitive. For $\succsim^*$ to be comparable, for a $B\in\mathcal{B}$ and $x,y\in B$, we must have either $x\in C(B)$, $y\in C(B)$ or both.