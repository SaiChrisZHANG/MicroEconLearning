Now we have two major approaches of decision making process: preference relations in Section \ref{chap1:sec1} and choice rules in Section \ref{chap1:sec2},
what we need to do is to link them. This linkage will emerge when we examine two central assumptions: \textbf{rationality} and \textbf{WARP}. So the major question here is: 
$$\textbf{rational}\succsim\xLeftrightarrow{???} (\mathcal{B},C(\cdot))\textbf{ satisfies WARP}$$
And the answer is: \textit{YES!} but not exactly. Now let's dig in.

\subsection*{Rational $\succsim\Rightarrow(\mathcal{B},C(\cdot))$ satisfies WARP}

First, $\textbf{rational}\succsim \Rightarrow (\mathcal{B},C(\cdot))\textbf{ satisfies WARP}$ is a big YES. To prove this, we need to define \textit{induced choice correspondence}:
\begin{definition}
    Given a \textbf{rational} $\succsim$ on $X$, if the decision maker faces a nonempty subset of alternatives $B\subset X$, by maximizing her preference, she would choose any one of the elements in the 
    \textit{induced choice correspondence}: $C^*(B,\succsim)=\{x\in B:x\succsim y, \forall y\in B\}$
\end{definition}

The induced choice correspondence $C^*(B,\succsim)$ has an important property: 
\begin{theorem}
    if $X$ is finite, $C^*(B,\succsim)$ will be \textbf{nonempty}.
\end{theorem}

A brief proof of this proposition is: If $X$ is finite, $B$ is finite as well. We will prove by induction. Starting from $|B|=1$, the only element of $B$ is in $C^*(B,\succsim)$. Now suppose $C^*(B,\succsim)$ is nonempty when $|B_n|=n$, 
let $x^*\in C^*(B_n,\succsim)$; when $|B_{n+1}|=n+1$, let the $n+1$th element $y$ $(\{y\}=B_{n+1}\setminus B_n)$. By the completeness of a rational $\succsim$, either $y\succsim x^*$ or $x^*\succsim y$:
\begin{enumerate}
    \item[i.] $y\succsim x^*$: since $x^*\in C^*(B_n,\succsim)\Rightarrow x^*\succsim x, \forall x\in B_n$. By transitivity of $\succsim$, $y\succsim x,\forall\in B_n$. By completeness, $y\succsim y$ as well. Hence, $y\in C^*(B_{n+1},\succsim)$.
    \item[ii.] $x^*\succsim y$: since $x^*\in C^*(B_n,\succsim)\Rightarrow x^*\succsim x, \forall x\in B_n$, hence $x^*\succsim x, \forall x\in B_n\cup{y}\Rightarrow x^*\in C^*(B_{n+1},\succsim)$
\end{enumerate}

Notice that when $B$ is finite, a stronger condition of $\succsim$ being acyclic and complete is equilavent to an induced choice rule $C^*(B,\succsim)\neq \varnothing$: 
\begin{theorem}\label{theo_acyclic_choice}
    For a finite $B$, $\succsim$ is complete and \textbf{acyclic} $\Leftrightarrow C^*(B,\succsim)\neq \varnothing$
\end{theorem}
$\succsim$ is acyclic mean that: $b_1\succsim b_2,b_2\succsim b_3,\cdots, b_{n-1}\succsim b_n\Rightarrow b_n\not\succsim b_1$. An example of transitive but not \textit{acyclic} relations is indifference $\sim$: $a_1\sim a_2\sim \cdots\sim a_n\Rightarrow a_n\sim a_1$.
A brief proof of Theorem \ref{theo_acyclic_choice} is:
\begin{enumerate}
    \item[i.] acyclic $\succsim\Rightarrow C^*(B,\succsim)\neq\varnothing$: Suppose if $C^*(B,\succsim)=\varnothing$, for $b_1\in B$, $b_1\notin C^*(B,\succsim)\Rightarrow \exists b_2$ s.t. $b_2\succsim b_1$. Continue this process, we can generate a sequence of $\cdots\succsim b_2\succsim b_1$, since $B$ is finite, this sequence must end at $b_n$. If $\succsim$ is acyclic, $b_1\not\succsim b_n$, this gives $b_n\succ b_1$, which would mean $b_n$ must be in $C^*(B,\succsim)$.
\end{enumerate}

With induced choice correspondence $C^*(B,\succsim)$ defined and non-emptyness proved, we can then say:
\begin{theorem}
    If $\succsim$ is a rational preference relation, then the choice structure generated by $\succsim$, $(\mathcal{B},C^*(\cdot,\succsim))$, satisfies WARP
\end{theorem}
