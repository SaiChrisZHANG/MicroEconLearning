In this section, I discuss some of common commentaries on the standard preference model presented above.

\subsection{Preference model as a descriptive model}
A common complaint about the standard utility maximization/preference ranking model is that no one in reality actually calculates a number as utility before making choices.
This comment has a lot of sense to it since we rarely care about utility, let alone doing some math, before grocery shopping. But this observation does NOT invalidate the 
usefulness of preference/utility model.

The standard model does NOT regulate agents to consciously maximize utility, instead, it assumes individuals act \textit{as if} they maximize utility. Mathematically, we have already
proven that if choice behavior satisfies finite nonemptiness and WARP, then something will be chosen, and agents' choice behavior is just \textit{as if} it were preference driven, or the choice
behavior can be linked to a preference. If the set of choices is countable, then the preference-driven choice can be indexed by numbers, hence, becomes a mathematical question.

Utility/preference/choice system is considered as a description of choice behavior. Long as people do make a choice, and that choice satisfies WARP, we can always find a numerical way to 
\textit{describe} the behavioral pattern.

\subsection{Empirical limits}
To verify utility maximization as a model of choices over the choice space $X$, we need to check every subset $A$ of it. And we also need to know all of $C(A)$. Of course we have already managed to verify the two-way link between
preference, utility and choice for all choice menus with no more than 3 elements, but above that, it would be extremely difficult. Empirically, we will observe (at best) $C(A)$ for finitely many subsets of $X$, we would most likely
observe only the \textit{one} element that is selected out of $C(A)$ while failing to identify equally-preferred alternatives simply because they are not observed to be chosen. In these scenarios, how can we tell whether our observations
are aligned with utility maximization? In later chapters, we will come back to this problem.

\subsection{Framing}
The way bundles are framed/presented can affect how they are perceived, hence influce individuals' decision making process. One of the most cited economic research by \citet{kai1979prospect} discussed this problem in a very clear and
innovative fashion. Framing will be a problem if it induces violence of WARP: $a$ is picked when comparing to $b$, but when $c$ is available as well, $b$ will be chosen instead. This may look silly and will never happen in real life but 
numerous examples of violation of WARP can be raised due to the framing problem. Designers/publicists are actually trained to take advantage of this "irrationality" to influence consumers' decision making process. A hugh strand of literature
in behavioral economics discuss and explore the framing problem, limited attention, heuristics, impatience are introduced to explain such phenomena. 

\subsection{Indecision}
Another big problem is that agents may just NOT be able to make a decision. Sometimes the differences between alternatives are trivial or too complicated to measure, the problem of indecision could rise. Rational preferences gives that for each pair of objects $x$ and $y$, an agent can 
choose between: $x$ is better than $y$, $y$ is better than $x$, $x$ and $y$ are equal. However, if we add another option \textit{I can't tell which is better}, the transitivity would be violated quite easily. Consider it this way, the choice of "I can't decide" allows $C(A)=\varnothing$ even
for a finite set of alternatives, this simply goes against the model we have built up.

\subsection{Inconsistency and probabilistic choice}
It is widely documented that an agent could be inconsistent about her choices. This could be an issue of framing, anchoring, indecision, or just unjustifiable inconsistency. This brings the stochastic side of choices: agents' choices may be subject to many random factors
hence not deterministic. This will be discussed more thoroughly in later chapters.

\subsection{Determinants of preference}
Since the model is a description of choice behavior, it does not provide any intuition on how a decision is made and what are the determinants of preferences. Later, we will talk about dynamic choice, where an agent's experiences affect her subsequent choices.
The standard model needs to be adjusted to incorporating the evolving decision making process through time. Another situation is welfare analysis. Institutional factors would need to be included in the models: preferences could be partially determined socially,
different social classes, countries, religions, cultures will likely lead agents to have different preferences. This has been examined more and more by institiutional economists, I will include some inspiring works later.