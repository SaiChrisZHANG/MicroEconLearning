In this section, I discuss some of common commentaries on the standard preference model presented above.

\subsection{Preference model as a descriptive model}
A common complaint about the standard utility maximization/preference ranking model is that no one in reality actually calculates a number as utility before making choices.
This comment has a lot of sense to it since we rarely care about utility, let alone doing some math, before grocery shopping. But this observation does NOT invalidate the 
usefulness of preference/utility model.

The standard model does NOT regulate agents to consciously maximize utility, instead, it assumes individuals act \textit{as if} they maximize utility. Mathematically, we have already
proven that if choice behavior satisfies finite nonemptiness and WARP, then something will be chosen, and agents' choice behavior is just \textit{as if} it were preference driven, or the choice
behavior can be linked to a preference. If the set of choices is countable, then the preference-driven choice can be indexed by numbers, hence, becomes a mathematical question.

Utility/preference/choice system is considered as a description of choice behavior. Long as people do make a choice, and that choice satisfies WARP, we can always find a numerical way to 
\textit{describe} the behavioral pattern.

\subsection{Empirical limits}
To verify utility maximization as a model of choices over the choice space $X$, we need to check every subset $A$ of it. And we also need to know all of $C(A)$. Of course we have already managed to verify the two-way link between
preference, utility and choice for all choice menus with no more than 3 elements, but above that, it would be extremely difficult. Empirically, we will observe (at best) $C(A)$ for finitely many subsets of $X$, we would most likely
observe only the \textit{one} element that is selected out of $C(A)$ while failing to identify equally-preferred alternatives simply because they are not observed to be chosen. In these scenarios, how can we tell whether our observations
are aligned with utility maximization? In later chapters, we will come back to this problem.

\subsection{Framing}
