We start from the basic: \textit{weak preference relation}, $\succsim$.
\begin{definition}\label{def_succsim}
    A weak preference relation $\succsim$ on a set $X$ is a subset of $X\times X$. If $(x,y)\in \succsim \Rightarrow$ $x$ is at least as good as $y$, written as $x\succsim y$
\end{definition}

A weak preference relation will induce two other types of relations on $X$:
\begin{definition}\label{def_succ_and_sim}
    With $\succsim$ defined by Def \ref{def_succsim}, we have
    \begin{enumerate}
        \item[-] the \textit{strict preference relation}, $\succ$ can be induced from $\succsim$ as: $x\succ y\Leftrightarrow x\succsim y \land y\not\succsim x$,
        or in words, $x$ if preferred to $y$.
        \item[-] the \textit{indifference relation}, $\sim$ can be induced from $\succsim$ as: $x\sim y \Leftrightarrow x\succsim y \land y\succsim x$, or in words, $x$ is indifferent to $y$.
    \end{enumerate}
\end{definition}

With the definition of these relations, we now define the central assumption of relations: \textit{\textbf{rationality}}. 
\begin{definition}
    A weak preference relation $\succsim$ is \textit{rational} if it is:
    \begin{enumerate}
        \item[-] Complete: $\forall x,y\in X$, $x\succsim y$ or $y\succsim x$ or both 
        \item[-] Transitive: $\forall x,y,z\in X$, $x\succsim y\land y\succsim z\Rightarrow x\succsim z$
    \end{enumerate}
\end{definition}

How to understand them? They are both strong assumptions:
\begin{enumerate}
    \item[-] Completeness of 
\end{enumerate}