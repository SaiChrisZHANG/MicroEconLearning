\minitoc

\vspace{0.5cm}
The first chapter summarizes the basic setting of individual decision making: preferences, choices and utilities. The main reference is Chapter 1 of \citet{mas1995microeconomic}.

In this chapter, we will focus on 3 domains: 
\begin{center}
    \begin{tabular}{rl}
    \hline
    \textbf{choice} & given a set $A$, what choice from $A$ is made \\ 
    \textbf{preference} & given alternatives $x,y$, which does the decision maker prefers \\ 
    \textbf{utility} & given an object $X$, how much does the DM likes $X$ (as a number)\\
    \hline
    \end{tabular}
\end{center}

The starting point of individual decision problem is a \textit{set of possible (mutually exclusive) alternatives} from which the individual must choose. To model decision making process
on this set of alternatives, one can:
\begin{enumerate}
    \item[-] either start from the tastes, i.e., \textit{preference relations} of individuals, and set up the patterns of decision making with preferences
    \item[-] or, start from the actual actions of individuals, i.e. \textit{choices}, to deduct a pattern of decision making
\end{enumerate}

With this two major approaches in mind, we know what's coming: the \textit{\textbf{rationality}} of preferences and the central assumption of choices, the \textbf{\textit{Weak Axiom of Revealed Preference (WARP)}}.
And of course, the two approaches and two basic assumptions are parallel, so we need to figure out how link the (underlying) preferences and (observed) choices.

\section{Preference Relations}\label{chap1:set_space}
We start from the basic: \textit{weak preference relation}, $\succsim$.
\begin{definition}\label{def_succsim}
    A weak preference relation $\succsim$ on a set $X$ is a subset of $X\times X$. If $(x,y)\in \succsim \Rightarrow$ $x$ is at least as good as $y$, written as $x\succsim y$
\end{definition}

A weak preference relation will induce two other types of relations on $X$:
\begin{definition}\label{def_succ_and_sim}
    With $\succsim$ defined by Def. \ref{def_succsim}, we have
    \begin{enumerate}
        \item[-] the \textit{strict preference relation}, $\succ$ can be induced from $\succsim$ as: $x\succ y\Leftrightarrow x\succsim y \land y\not\succsim x$,
        or in words, $x$ if preferred to $y$.
        \item[-] the \textit{indifference relation}, $\sim$ can be induced from $\succsim$ as: $x\sim y \Leftrightarrow x\succsim y \land y\succsim x$, or in words, $x$ is indifferent to $y$.
    \end{enumerate}
\end{definition}

With the definition of these relations, we now define the central assumption of relations: \textit{\textbf{rationality}}. 
\begin{definition}\label{def_succsim_rational}
    A weak preference relation $\succsim$ is \textit{rational} if it is:
    \begin{enumerate}
        \item[-] Complete: $\forall x,y\in X$, $x\succsim y$ or $y\succsim x$ or both 
        \item[-] Transitive: $\forall x,y,z\in X$, $x\succsim y\land y\succsim z\Rightarrow x\succsim z$
    \end{enumerate}
\end{definition}

How to understand them? They are both strong assumptions:
\begin{enumerate}
    \item[-] Completeness of $\succsim$ means it is well-defined between any two possible alternatives. From the perspective of an individual, completeness means that she will make choices, and only meditated choices.
    \item[-] Transitivity of $\succsim$ implies that the decision maker will not have a preference cycle, since whoever has a preference cycle would suffer economically for it\footnote{There are 2 types of violations of 
    transitivity: irrational and mechanical. Irrational violations are easy to understand: decision makers simply do not follow transivity assumption, many reasons have been raised, including mental account, framing, menu effect, attraction effect, etc. Mechanical violations means that decision makers are "forced" to violate transitivity. One 
    example of this type of violation is aggregation of considerations: decision makers may aggregate several sub-preferences as together to make the choice, leading to violation of transitivity. Another example is when the preference
    is only defined for differences above a certain level (problem of perceptible differences). See \citet[Page 7-8]{mas1995microeconomic}, \citet[Page 4-5]{ariel2012lecture} for details}.   
\end{enumerate}

With the definition of rational $\succsim$ in Def. \ref{def_succsim_rational}, we can prove the following properites of $\succ$ and $\sim$ \textit{induced} by $\succsim$:
\begin{theorem} If $\succsim$ is rational, then:
    \begin{enumerate}
        \item[i.] $\succ$ is irreflexive ($x\succ x$ never holds) and transitive ($x\succ y \land y\succ z\Rightarrow x\succ z$)
        
        \underline{Proof}:
        \begin{enumerate}
            \item[-] irreflexive: by Def. \ref{def_succ_and_sim}, $x\succ x\Rightarrow x\succsim x \land x\not\succsim x$, self contracdiction.
            \item[-] transitive: $x\succ y \Rightarrow x\succsim y \land y\not\succsim x$, $y\succ z \Rightarrow y\succsim z \land z\not\succsim y$. By transitivity of $\succsim$, $x\succsim y\land y\succsim z\Rightarrow x\succsim z$. If $z\succsim x$, by transitivity of $\succsim$ and $x\succsim y$, we would have $z\succsim y$, contradicting $y\succ z$. Therefore $x\succsim z \land z\not\succsim x\Rightarrow x\succ z$. 
        \end{enumerate}
        \item[ii.] $\sim$ is reflexive ($x\sim x, \forall x$), transitive ($x\sim y\land y\sim z\Rightarrow x\sim z$) and symmetric ($x\sim y\Rightarrow y\sim x$)
        
        \underline{Proof}:
        \begin{enumerate}
            \item[-] reflexive: by completeness of $\succsim$, $\forall x, x\succsim x\Rightarrow x\sim x$
            \item[-] transitive: $x\sim y\Rightarrow x\succsim y \land y\succsim x$, $y\sim z\Rightarrow y\succsim z, z\succsim y$, by the transitivity of $\succsim$, we have $x\succsim z \land z\succsim x$, hen $x\sim z$
            \item[-] symmetric: $x\sim y \Rightarrow x\succsim y \land y\succsim x\Leftrightarrow y\succsim x \land x\succsim y\Rightarrow y\sim x$
        \end{enumerate}
        \item[iii.] $x\succ y\succsim z\Rightarrow x\succ z$ 
        
        \underline{Proof}: $x\succ y\Rightarrow x\succsim y\land y\not\succsim x$, hence $x\succ y\succsim z\Rightarrow x\succsim z$. If $z\succsim x$, by transitivity of $\succsim$, $y\succsim x$, contradicting $x\succ y$. Therefore, $z\not\succsim x$
    \end{enumerate}
\end{theorem}

We can also directly define a \textit{rational} $\succ$ (see \citet[Page 19-21]{kreps1990acourse}):
\begin{definition}\label{def_rational_succ}
    A strict preference ralation $\succ$ is rational if it is:
    \begin{enumerate}
        \item[-] asymmetric: $\nexists x,y\in X$ s.t. $x\succ y \land y\succ x$
        \item[-] negatively transitive: $x\succ y \Rightarrow \forall z\in X\setminus{x,y}$, $x\succ z\lor z\succ y \lor$both. 
    \end{enumerate}
\end{definition}

With Def. \ref{def_rational_succ} and Def. \ref{def_succsim_rational}, we can prove that $\succsim$ is rational iff $\succ$ is rational:
\begin{theorem}
    $\succsim$ is rational $\Leftrightarrow$ $\succ$ is rational, specifically:
    \begin{enumerate}
        \item[-] $\succsim$ is complete $\Leftrightarrow$ $\succ$ is asymmetric
        \item[-] $\succsim$ is transitive $\Leftrightarrow$ $\succ$ is negatively transitive
    \end{enumerate}
\end{theorem}

Now we prove this theorem:
\begin{description}
    \item[Step 1] proof $\succsim$ is rational $\Rightarrow$ $\succ$ 
    \begin{itemize}
        \item[-] \textbf{asymmetric}
        
        if $\exists x,y$ s.t. $x\succ y$ and $y\succ x$, then by the definition of induced strict preference, the pair $x,y$ must satisfy 
        $$\begin{cases}
        x\succsim y\text{ and }y\not\succsim x & (x\succ y)\\
        y\succsim x\text{ and }x\not\succsim y & (y\succ x)
        \end{cases}$$
        which is, by completeness of rational $\succsim$, impossible. Therefore, such pair $x,y$ don't exist. $\succ$ is proved to be asymetric.
        
        \item[-] \textbf{negatively transitive}
        
        First, $\forall z\notin\{x,y\}$, by completeness of rational $\succsim$, the relation between $x$ and $z$ is either $x\succsim z$ or $z\succsim x$.
        Similarly, the relation between $y$ and $z$ is either $y\succsim z$ or $z\succsim y$.
        
        Second, given $x\succ y$, $x,y$ satisfies $x\succsim y$ and $y\not\succsim x$.
        
        Also, it is easy to prove that: $x\succ y \land y\succsim z \Rightarrow x\succ z$, $x\succ y \land z\succsim x\Rightarrow z\succ y$; and $x\succ y\land z\sim x\Rightarrow z\succ y$, $x\succ y\land y\sim z\Rightarrow x\succ z$

        Now we have the following scenarios:
        \begin{enumerate}
        \item if $z\succsim x$ and $y\succsim z$, by transitivity of rational
        $\succsim$, $y\succsim x$, contradicting the definition of $x\succ y$.
        This scenario doesn't exist.
        \item if $x\succsim z$ and $y\succsim z$, since $x\succ y$, with the
        auxilary result proved above, we have $x\succ z$
        \item if $z\succsim x$ and $z\succsim y$, since $x\succ y$, with the
        auxilary result proved above, we have $z\succ y$
        \item if $x\succsim z$ and $z\succsim y$, since $x\succ y$, suppose:
        \begin{enumerate}
        \item $z\succsim x$ as well, then $x\sim z$, in this case $z\succ y$;
        \item $z\not\succsim x$, then $x\succ z$
        \item $y\succsim z$ as well, then $y\sim z$, in this case $x\succ z$
        \item $y\not\succsim z$, then $z\succ y$
        \end{enumerate}
        therefore, a complete summary of (a) to (d) would give:
        
        \begin{table}[h]
        \begin{centering}
        \begin{tabular}{ccc}
        \hline 
         & $z\succsim x$ & $z\not\succsim x$\tabularnewline
        \hline 
        $y\succsim z$ & $z\succ y$ \& $x\succ z$ & $x\succ z$\tabularnewline
        $y\not\succsim z$ & $z\succ y$ & $x\succ z$ \& $z\succ y$\tabularnewline
        \hline 
        \end{tabular}
        \par\end{centering}
        \end{table}
        
        \end{enumerate}
        Combining all above, we have proved negative transitivity of $\succ$.
        \end{itemize}
        With asymmetry and negative transitivity proved, we've proved that
        $\succsim$ is rational $\Rightarrow$$\succ$ is rational
    
    \item[Step 2] proof $\succ$ is rational $\Rightarrow$$\succsim$ is
    rational. Again, the proof is done separately for the two properties.
    \begin{itemize}
    \item[-] Complete: with a rational $x\succ y$, we know $\nexists x,y$ s.t.
    $x\succ y$ and $y\succ x$ by asymmetry. Therefore, $\forall x,y$,
    we have two possibilities.
    \begin{itemize}
    \item $x\succ y$ and $y\not\succ x$, which would naturally induce a weak
    preference $x\succsim y$
    \item $y\succ x$ and $x\not\succ y$, which would naturally induce a weak
    preference $y\succsim x$
    \end{itemize}
    therefore, $\forall x,y$, either $x\succsim y$ or $y\succsim x$ completeness of $\succsim$ is proven.

    \item[-] Transitive: with a rational $x\succ y$, negative transivity gives
    $\forall z\notin\{x,y\},$ either $x\succ z$, $z\succ y$, or both.
    By negative transitivity, we have:
    \begin{itemize}
    \item $x\succ z$: following same procedure, we know $x\succsim z$. If:
    \begin{itemize}
    \item $y\succsim z$, since $x\succ z\Rightarrow z\not\succsim x$, by completeness
    we have $x\succsim z$, thus $x\succsim y\land y\succsim z\Rightarrow x\succsim z$
    \item $z\succsim y$, since $x\succ y\Rightarrow x\not\succsim y$, by completeness
    we have $x\succsim y$, thus $x\succsim z\land z\succsim y\Rightarrow x\succsim y$
    \end{itemize}
    \item $z\succ y$: again, we know $z\succsim y$. If:
    \begin{itemize}
    \item $x\succsim z$, since $x\succ y\Rightarrow y\not\succsim x$, by completeness
    we have $x\succsim y$, thus $z\succsim y\land x\succsim z\Rightarrow x\succsim y$
    \item $z\succsim x$, with $x\succsim y$, suppose $y\succsim z$, this
    contradicts $z\succ y$, thus $z\succsim x\land x\succsim y\Rightarrow z\succsim y$
    \end{itemize}
    \item $x\succ z$ and $z\succ y$: again we know $x\succsim z$ and $z\succsim y$.
    Suppose $y\succsim x$, this contradicts $x\succ y$, therefore $x\succsim z\land z\succsim y\Rightarrow x\succsim y$
    \end{itemize}
    In all three scenarios, transitivity is proved.
    \end{itemize}
    With completeness and transitivity proved, we've proved that $\succ$
    is rational $\Rightarrow\succsim$ is rational.
        
\end{description}

We can do

\section{Choice Rules}\label{chap1:sec2}
Next, we approach the theory of decision making from choice behavior itself. Formally, choice behavior is represented by means of a \textit{choice structure} $(\mathcal{B},C(\cdot))$. Now, we define choice structure $(\mathcal{B},C(\cdot))$:
\begin{definition}
    A choice structure $(\mathcal{B},C(\cdot))$ has two ingredients:
    \begin{enumerate}
        \item[-] $\mathcal{B}\subset \mathcal{P}(X)\setminus\varnothing$, where $\mathcal{P}(X)$ is the power set of $X$. This means, every element $B\in \mathcal{B}$ is a subset of $X$\footnote{The elements $B\in\mathcal{B}$ are so-called \textit{budget sets}. The budget sets in $\mathcal{B}$ should be thought of as an exhaustive listing of all the choice experiments that can be achieved, but it is possible that some subsets of $X$ are not achievable.}.
        \item[-] $C(\cdot)$ is a \textit{choice rule correspondence} that assigns a nonempty set of chosen elements $C(B)\subset B$, $\forall B\in \mathcal{B}$\footnote{The choice set $C(B)$ can contain a single element, which is the choice among the alternatives in $B$. BUT, $C(B)$ can contain multiple elements, then elements of $C(B)$ are the \textit{acceptable alternatives} in $B$.}.
    \end{enumerate}
\end{definition}

Now we discuss the CORE assumption in this section: the Weak Axiom of Revealed Preference (WARP):
\begin{definition}
    A choice set $(\mathcal{B},C(\cdot))$ satisfies WARP if:
    \begin{enumerate}
        \item[-] $\forall B,B^{\prime}$ and $x,y\in B\cap B^{\prime}$, $x\in C(B),y\in C(B^{\prime})\Rightarrow x\in C(B^{\prime})$
    \end{enumerate}
\end{definition}

Or in words, WARP requires that if $x$ is chosen from some alternatives where $y$ is also available, then there can be NO budget set containing both $x$ and $y$
but only $y$ is chosen.

Following WARP, define the \textit{reveal preference relation} $\succsim^*$ as:
\begin{definition}
    Given a choice structure $(\mathcal{B},C(\cdot))$, $x\succsim^*y\Leftrightarrow \exists B\in\mathcal{B}$ s.t. $x,y\in B\land x\in C(B)$
\end{definition}
In words, $x$ is revealed at least as good as $y$.

With revealed preference defiend, we can rephrase WARP as: \textit{If x is revealed at least as good as y, then y \textbf{cannot} be revealed preferred to x}. Hence, $\succsim^*$ is not symmetric.

One thing to remember is that $\succsim^*$ need not be either complete or transitive. For $\succsim^*$ to be comparable, for a $B\in\mathcal{B}$ and $x,y\in B$, we must have either $x\in C(B)$, $y\in C(B)$ or both.

\section{Linking Preferences with Choices}\label{chap1:sec3}

\section{Chap1Sec4}\label{chap1:sec4}

