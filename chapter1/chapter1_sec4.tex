Now, with preferences and choices defined, and the linkage between the two established, we need to transfer these concepts into math for analytic studies.
This is exactly why utility functions are introduced: to assign a number and rank the elements in $X$ according to preferences.

\begin{definition}
    A function $u:X\rightarrow\mathbb{R}$ is a \textit{utility function representing relation} $\succsim$ if $\forall x,y\in X,x\succsim y\Leftrightarrow u(x)\geq u(y)$
\end{definition}

Notice that a utility function representing a preference relation $\succsim$ is NOT unique. \textbf{Rank-preserving} is the only requirement, hence, any 
strictly increasing function $f:\mathbb{R}\rightarrow\mathbb{R},v(x)=f(u(x))$ will also represent $\succsim$ as $u(\cdot)$.