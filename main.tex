\documentclass[12pt,openany]{report}
\setlength{\headheight}{52pt}
\usepackage[a4paper,left=1in,right=1in,top=1in,bottom=1in, heightrounded, marginparwidth=2cm, marginparsep=3mm]{geometry}
\usepackage[utf8x]{inputenc}
\usepackage[english]{babel}
\setlength{\parskip}{0.5em}
\usepackage{url}
\usepackage{titlesec}
\setcounter{secnumdepth}{4}
\usepackage{palatino}
\usepackage{tocloft}
\usepackage[nottoc]{tocbibind}
\usepackage{marginnote}
\usepackage{multirow}
\usepackage{easybmat,bigdelim,arydshln}
\usepackage[authoryear,round]{natbib}
\usepackage{amssymb,amsmath,amsthm,amsfonts}
\usepackage{mathtools}
\usepackage{graphicx}
\graphicspath{{./chapter1/outputs/}}
\usepackage{hyperref}
\usepackage{tcolorbox}
\usepackage{newpxtext,newpxmath}
\usepackage{fancyhdr}
\usepackage[Conny]{fncychap}
    \ChNameVar{\centering\huge\fontfamily{ppl}\selectfont\rm\bfseries}
    \ChNumVar{\huge\fontfamily{ppl}\selectfont}
    \ChTitleVar{\centering\LARGE\fontfamily{ppl}\selectfont\rm}

\usepackage{xcolor}

\hypersetup{
    colorlinks,
    citecolor=red,
    filecolor=black,
    linkcolor=violet,
    urlcolor=blue
}

\makeatletter
\newtheoremstyle{indented}
  {3pt}% space before
  {3pt}% space after
  {\addtolength{\@totalleftmargin}{3.5em}
   \addtolength{\linewidth}{-3.5em}
   \parshape 1 3.5em \linewidth}% body font
  {}% indent
  {\bfseries}% header font
  {.}% punctuation
  {.5em}% after theorem header
  {}% header specification (empty for default)
\makeatother

\theoremstyle{indented}
\newtheorem{definition}{Definition}
\newtheorem{theorem}{Theorem}
\newtheorem{example}{Example}
\numberwithin{definition}{section}
\numberwithin{theorem}{section}
\numberwithin{example}{section}

\newcommand*\diff{\mathop{}\!\mathrm{d}}
\newcommand*\mdiff{\mathop{}\!\mathrm{D}}
\newcommand*\Diff[1]{\mathop{}\!\mathrm{d^#1}}

\renewcommand\cftchapafterpnum{\vskip5pt}
\renewcommand\cftsecafterpnum{\vskip0pt}

\fancypagestyle{mystyle}{%
  \fancyhf{}% Clear header/footer
  \renewcommand{\headrulewidth}{1pt}% Remove header rule
  \renewcommand{\footrulewidth}{1pt}% Remove footer rule
  \fancyhead[R]{\href{https://github.com/SaiChrisZHANG/MicroEconLearning}{Github Page}}
  \fancyhead[L]{Empirical Finance: A Review}
  \fancyfoot[L]{\hyperref[ToC-first-page]{To Contents}}
  \fancyfoot[C]{\small \leftmark}
  \fancyfoot[R]{\thepage}
}

\usepackage{minitoc}
\newcommand{\sidenotes}[1]{\marginnote{\raggedright\scriptsize#1}}

\usepackage{paralist}
  \let\itemize\compactitem
  \let\enditemize\endcompactitem
  \let\enumerate\compactenum
  \let\endenumerate\endcompactenum
  \let\description\compactdesc
  \let\enddescription\endcompactdesc
  \pltopsep=1pt
  \plitemsep=1pt
  \plparsep=1pt

\usepackage{etoolbox}
\AtBeginEnvironment{quote}{\vspace{1pt}}
\AtEndEnvironment{quote}{\vspace{1pt}}

\begin{document}

\begin{titlepage}
    \begin{center}
        \vspace*{1cm}
        
        \Huge
        \textbf{Microeconomic Theory I: A Notebook}

        \Large
        \textit{With Jonathan Libgober}
            
        \vspace{2.5cm}
        
        \LARGE    
        \textbf{Sai Zhang}
            
        \vfill
        
        \large    
        Check the \href{https://github.com/SaiChrisZHANG/MicroEconLearning}{Github Page} of this project, or \href{mailto:saizhang.econ@gmail.com}{email me}!

        \vspace{0.8cm}
        \large
        \today
            
    \end{center}
\end{titlepage}

%%%%%%%% Main content %%%%%%%%%%
%%%%%%%% Chapters are called separately %%%%%%%%%%

\chapter*{Here we go!}

This is my learning notebook of Microeconomic Theory I (Course number: ECON601 at USC Economics). As one of the core courses in 
an economic Ph.D. curriculum, Microeconomic Theory I is beyond important to my research. Therefore, I would love to use this notebook
as a commitment mechanism, to document lecture notes, discuss session and office hour intuitions, reading summaries, my personal questions
regarding the topics and more. By building a file from scratch, hopefull I could have a more systematic and sophisticated understanding on
the content of this course.

I thank Prof. Jonathan Libgober at USC Economics for leading the discussion of the course and providing intuitive ways to understand microeconomic theory.
Please check his webpage \href{https://www.jonlib.com/}{here}, he is such fun.

I also appreciate the time and effort my TA Qitong Wang put into this course, guiding me through discussing sessions and problem sets. When I have questions, he is 
always there to help.

Following the structure of the course, this notebook will cover three aspects of microeconomic theories: (a) individual decision making, (b) game theory, (c) mechanism design and contract theory. 
Apart from Jonathan's lecture notes, I will also summarize the reading materials, including: 
\cite{mas1995microeconomic}'s \textit{Microeconomic Theory}, \citet{mailath2018modeling}'s \textit{Modelling Strategic Behavior}\footnote{Latest version (May 2021) available \href{https://web.sas.upenn.edu/gmailath/books/modeling-strategic-behavior/}{here}.},
\cite{tirole1991game}'s \textit{Game Theory}, \cite{myerson1991game}'s \textit{Game Theory: Analysis of Conflicts}, \cite{bolton2005contract}'s \textit{Contract Theory}, 
\cite{Mailath2006repeated}'s \textit{Repeated Games and Reputation} and \cite{martin1994acourse}'s \textit{A Course in Game Theory}. Other materials will also be referred to along the way.


Building this notebook is truly a memorable journey for me. I would love to share this review and all the related
materials to anyone that finds them useful. And unavoidably, I would make some
typos and other minor mistakes (hopefully not big ones). So I'd really appreciate
any correction. If you find any mistakes, please send the mistakes to this email address \href{mailto:saizhang.econ@gmail.com}{saizhang.econ@gmail.com} or start a branch on \href{https://github.com/SaiChrisZHANG/MicroEconLearning}{Github}. BIG thanks in advance!

\newpage

\dominitoc
\phantomsection
\label{ToC-first-page}
\tableofcontents

\pagestyle{mystyle}
\part{Individual Decision Making}

\chapter{Preferences and Choices, Utilities}
\minitoc

\vspace{0.5cm}
The first chapter summarizes the basic setting of individual decision making: preferences, choices and utilities. The main reference is Chapter 1 of \citet{mas1995microeconomic}.

The starting point of individual decision problem is a \textit{set of possible (mutually exclusive) alternatives} from which the individual must choose. To model decision making process
on this set of alternatives, one can:
\begin{enumerate}
    \item[-] either start from the tastes, i.e., \textit{preference relations} of individuals, and set up the patterns of decision making with preferences
    \item[-] or, start from the actual actions of individuals, i.e. \textit{choices}, to deduct 
\end{enumerate}
either ; 

\section{Set Theory and Space Theory}\label{chap1:set_space}
We start from the basic: \textit{weak preference relation}, $\succsim$.
\begin{definition}\label{def_succsim}
    A weak preference relation $\succsim$ on a set $X$ is a subset of $X\times X$. If $(x,y)\in \succsim \Rightarrow$ $x$ is at least as good as $y$, written as $x\succsim y$
\end{definition}

A weak preference relation will induce two other types of relations on $X$:
\begin{definition}
    With $\succsim$ defined by Def \ref{def_succsim}, we have
    \begin{enumerate}
        \item[-] the \textit{strict preference relation}, $\succ$ can be induced from $\succsim$ as: $x\succ y\Leftrightarrow x\succsim y \wedge y\not\succsim x$,
        or in words, $x$ if preferred to $y$.
        \item[-] the \textit{indifference relation}, $\sim$ can be induced from $\succsim$ as: $x\sim y \Leftrightarrow x\succsim y \wedge y\succsim x$, or in words, $x$ is indifferent to $y$.
    \end{enumerate}
\end{definition}

\section{Chap1Sec2}\label{chap1:sec2}
Next, we approach the theory of decision making from choice behavior itself. Formally, choice behavior is represented by means of a \textit{choice structure} $(\mathcal{B},C(\cdot))$. Now, we define choice structure $(\mathcal{B},C(\cdot))$:
\begin{definition}
    A choice structure $(\mathcal{B},C(\cdot))$ has two ingredients:
    \begin{enumerate}
        \item[-] $\mathcal{B}\subset \mathcal{P}(X)\setminus\varnothing$, where $\mathcal{P}(X)$ is the power set of $X$. This means, every element $B\in \mathcal{B}$ is a subset of $X$\footnote{The elements $B\in\mathcal{B}$ are so-called \textit{budget sets}. The budget sets in $\mathcal{B}$ should be thought of as an exhaustive listing of all the choice experiments that can be achieved, but it is possible that some subsets of $X$ are not achievable.}.
        \item[-] $C(\cdot)$ is a \textit{choice rule correspondence} that assigns a nonempty set of chosen elements $C(B)\subset B$, $\forall B\in \mathcal{B}$\footnote{The choice set $C(B)$ can contain a single element, which is the choice among the alternatives in $B$. BUT, $C(B)$ can contain multiple elements, then elements of $C(B)$ are the \textit{acceptable alternatives} in $B$.}.
    \end{enumerate}
\end{definition}

Now we discuss the CORE assumption in this section: the Weak Axiom of Revealed Preference (WARP):
\begin{definition}
    A choice set $(\mathcal{B},C(\cdot))$ satisfies WARP if:
    \begin{enumerate}
        \item[-] $\forall B,B^{\prime}$ and $x,y\in B\cap B^{\prime}$, $x\in C(B),y\in C(B^{\prime})\Rightarrow x\in C(B^{\prime})$
    \end{enumerate}
\end{definition}

Or in words, WARP requires that if $x$ is chosen from some alternatives where $y$ is also available, then there can be NO budget set containing both $x$ and $y$
but only $y$ is chosen.

Following WARP, define the \textit{reveal preference relation} $\succsim^*$ as:
\begin{definition}\label{def_revealed_pref}
    Given a choice structure $(\mathcal{B},C(\cdot))$, $x\succsim^*y\Leftrightarrow \exists B\in\mathcal{B}$ s.t. $x,y\in B\land x\in C(B)$
\end{definition}
In words, $x$ is revealed at least as good as $y$.

With revealed preference defiend, we can rephrase WARP as: \textit{If x is revealed at least as good as y, then y \textbf{cannot} be revealed preferred to x}. Hence, $\succsim^*$ is not symmetric.

One thing to remember is that $\succsim^*$ need not be either complete or transitive. For $\succsim^*$ to be comparable, for a $B\in\mathcal{B}$ and $x,y\in B$, we must have either $x\in C(B)$, $y\in C(B)$ or both.

An example is:
\begin{example}
    Consider a choice structure $(\mathcal{B},C(\cdot))$ from $X=\{x,y,z\}$, where $\mathcal{B}=\{\{x,y\},\{x,y,z\}\}$. Under WARP, $C\{x,y\}=\{x\}\Rightarrow y\notin C\{x,y,z\}$.
    BUT, we can have $z\in C(\{x,y,z\})$.
\end{example}

This is why the induced preference is called \textit{revealed}: you don't know what else is going on.

\section{Chap1Sec3}\label{chap1:sec3}

\section{Chap1Sec4}\label{chap1:sec4}



\chapter{Fundamentals of Consumer Theory}
\minitoc

\vspace{0.5cm}
The second chapter focuses on the most fundamental decision unit of microeconomic theory: \textit{consumer}. The main reference is Chapter 2 and 3 of \citet{mas1995microeconomic}.

The basic setting of consumer demand study is \textit{market economy}, where the goods and services that the consumer may acquire and consume are available for purchase at known prices (or trade for other goods at know exchange rates).

In this chapter, we will focus on 2 major aspects of the consumer theory: choice and demand.
\begin{center}
    \begin{tabular}{rl}
    \hline
    \textbf{choice} & individual decision making analysis based on choice\\ 
    \textbf{demand} & individual decision making analysis based on preference \\ 
    \hline
    \end{tabular}
\end{center}

The starting point of individual decision problem is a \textit{set of possible (mutually exclusive) alternatives} from which the individual must choose. To model decision making process
on this set of alternatives, one can:
\begin{enumerate}
    \item[-] either start from the tastes, i.e., \textit{preference relations} of individuals, and set up the patterns of decision making with preferences
    \item[-] or, start from the actual actions of individuals, i.e. \textit{choices}, to deduct a pattern of decision making
\end{enumerate}

The two aspects of consumer theory are actually closely related to each other. Just like choices and preferences in Chapter 1, they are two sides of the same coin. However, they are NOT equivalent.
The major conclusion of choice-based consumer theory is WARP is essentially equivalent to the \textit{compensated law of demand}, but WARP imposes fewer restrictions on demand than preference-based theory,
hence, does NOT necessarily guarantee the existence of a rationalizing preference relation for consumer demand, therefore, \textit{strong axiom of revealed preference} is introduced.

\section{Basic setting}\label{chap2:sec1}
First, we introduce the basic settings of a consumer's problem in a market economy. These concepts will keep reoccuring in the following sections.

\subsection{Commodities}
First, we need to define the goods and services the consumers consume. We do not actually care about what they specifically are, instead, we use a very abstract concept \textit{commodities}
to summarize and analyze them.
\begin{definition}
    Assume there are $L<\infty$ different commodities, a \textit{commodity vector} or \textit{commodity bundle} is a list of amounts of the different commodities:$$x=\left[x_1,\cdots,x_L\right]^T$$
\end{definition}
$x$ can be view as a point in an $\mathbb{R}^L$ space, i.e., the commodity space. Each entry $x_l$ of $x$ ($l=1,\cdots,L$) represents the amount of commodity $l$ consumed, hence, the vector is referred to
as \textit{consumption vector} or \textit{consumption bundle}.

Three things to keep in mind:
\begin{enumerate}
    \item[-] Time can be incorporated into this setting, namely, today's commodity is distinct from tomorrow's commodity, even if they are otherwise the same. The value of time will come back in later chapters and is crucial in a large strand of behavioral economic literature. Same logic applies to other limitations that are easily neglected, like geographic ones.
    \item[-] Negative entries can exists in a commodity vector, indicating debits or net outflows of goods. In a producing problem or exchange problem, negative entries of commodity vectors are not rare.
    \item[-] Consumption is quite flexible and comes in many format empirically, for the sake of data collection conveniency, consumption data are often aggregated monthly, quarterly for even annually. Meanwhle, some consumptions in the commodity vectors may not actually occur in the market.
\end{enumerate}

\subsection{The Consumption Set}
Consumptions are limited by a number of constraints, which will form a subset of commodity space $X\subset \mathcal{R}^L$. With in this subset, all possible commodity bundles can be 
consumed, this is exactly the definition of consumption sets (see \citet[Page 19-20]{mas1995microeconomic} for some simple examples of consumption sets).

For now, we will focus on the simplest consumption set: all possible non-negative commodity bundles:
$$X=\mathbb{R}^L_+=\{x\in \mathbb{R}^L: x_l\geq 0,\forall l=1,\cdots, L\}$$

It is easy to show that
$$\mathbb{R}^L_+ \text{ is a }\textbf{convex} \text{ set}$$

A brief proof: $\forall x,x'\in \mathbb{R}^L_+$

\section{Choice Rules}\label{chap2:sec2}
%Next, we approach the theory of decision making from choice behavior itself. Formally, choice behavior is represented by means of a \textit{choice structure} $(\mathcal{B},C(\cdot))$. Now, we define choice structure $(\mathcal{B},C(\cdot))$:
\begin{definition}
    A choice structure $(\mathcal{B},C(\cdot))$ has two ingredients:
    \begin{enumerate}
        \item[-] $\mathcal{B}\subset \mathcal{P}(X)\setminus\varnothing$, where $\mathcal{P}(X)$ is the power set of $X$. This means, every element $B\in \mathcal{B}$ is a subset of $X$\footnote{The elements $B\in\mathcal{B}$ are so-called \textit{budget sets}. The budget sets in $\mathcal{B}$ should be thought of as an exhaustive listing of all the choice experiments that can be achieved, but it is possible that some subsets of $X$ are not achievable.}.
        \item[-] $C(\cdot)$ is a \textit{choice rule correspondence} that assigns a nonempty set of chosen elements $C(B)\subset B$, $\forall B\in \mathcal{B}$\footnote{The choice set $C(B)$ can contain a single element, which is the choice among the alternatives in $B$. BUT, $C(B)$ can contain multiple elements, then elements of $C(B)$ are the \textit{acceptable alternatives} in $B$.}.
    \end{enumerate}
\end{definition}

Now we discuss the CORE assumption in this section: the Weak Axiom of Revealed Preference (WARP):
\begin{definition}
    A choice set $(\mathcal{B},C(\cdot))$ satisfies WARP if:
    \begin{enumerate}
        \item[-] $\forall B,B^{\prime}$ and $x,y\in B\cap B^{\prime}$, $x\in C(B),y\in C(B^{\prime})\Rightarrow x\in C(B^{\prime})$
    \end{enumerate}
\end{definition}

Or in words, WARP requires that if $x$ is chosen from some alternatives where $y$ is also available, then there can be NO budget set containing both $x$ and $y$
but only $y$ is chosen.

Following WARP, define the \textit{reveal preference relation} $\succsim^*$ as:
\begin{definition}\label{def_revealed_pref}
    Given a choice structure $(\mathcal{B},C(\cdot))$, $x\succsim^*y\Leftrightarrow \exists B\in\mathcal{B}$ s.t. $x,y\in B\land x\in C(B)$
\end{definition}
In words, $x$ is revealed at least as good as $y$.

With revealed preference defiend, we can rephrase WARP as: \textit{If x is revealed at least as good as y, then y \textbf{cannot} be revealed preferred to x}. Hence, $\succsim^*$ is not symmetric.

One thing to remember is that $\succsim^*$ need not be either complete or transitive. For $\succsim^*$ to be comparable, for a $B\in\mathcal{B}$ and $x,y\in B$, we must have either $x\in C(B)$, $y\in C(B)$ or both.

An example is:
\begin{example}
    Consider a choice structure $(\mathcal{B},C(\cdot))$ from $X=\{x,y,z\}$, where $\mathcal{B}=\{\{x,y\},\{x,y,z\}\}$. Under WARP, $C\{x,y\}=\{x\}\Rightarrow y\notin C\{x,y,z\}$.
    BUT, we can have $z\in C(\{x,y,z\})$.
\end{example}

This is why the induced preference is called \textit{revealed}: you don't know what else is going on.

\section{Linking Preferences with Choices}\label{chap2:sec3}
%Now we have two major approaches of decision making process: preference relations in Section \ref{chap1:sec1} and choice rules in Section \ref{chap1:sec2},
what we need to do is to link them. This linkage will emerge when we examine two central assumptions: \textbf{rationality} and \textbf{WARP}. So the major question here is: 
$$\textbf{rational}\succsim\xLeftrightarrow{???} (\mathcal{B},C(\cdot))\textbf{ satisfies WARP}$$
And the answer is: \textit{YES!} but not exactly. Now let's dig in.

\subsection*{Rational $\succsim\Rightarrow(\mathcal{B},C(\cdot))$ satisfies WARP}

First, $\textbf{rational}\succsim \Rightarrow (\mathcal{B},C(\cdot))\textbf{ satisfies WARP}$ is a big YES. To prove this, we need to define \textit{induced choice correspondence}:
\begin{definition}\label{def_induced_choice}
    Given a \textbf{rational} $\succsim$ on $X$, if the decision maker faces a nonempty subset of alternatives $B\subset X$, by maximizing her preference, she would choose any one of the elements in the 
    \textit{induced choice correspondence}: $C^*(B,\succsim)=\{x\in B:x\succsim y, \forall y\in B\}$
\end{definition}

The induced choice correspondence $C^*(B,\succsim)$ has an important property: 
\begin{theorem}
    if $X$ is finite, $C^*(B,\succsim)$ will be \textbf{nonempty}.
\end{theorem}

A brief proof of this proposition is: If $X$ is finite, $B$ is finite as well. We will prove by induction. Starting from $|B|=1$, the only element of $B$ is in $C^*(B,\succsim)$. Now suppose $C^*(B,\succsim)$ is nonempty when $|B_n|=n$, 
let $x^*\in C^*(B_n,\succsim)$; when $|B_{n+1}|=n+1$, let the $n+1$th element $y$ $(\{y\}=B_{n+1}\setminus B_n)$. By the completeness of a rational $\succsim$, either $y\succsim x^*$ or $x^*\succsim y$:
\begin{enumerate}
    \item[i.] $y\succsim x^*$: since $x^*\in C^*(B_n,\succsim)\Rightarrow x^*\succsim x, \forall x\in B_n$. By transitivity of $\succsim$, $y\succsim x,\forall\in B_n$. By completeness, $y\succsim y$ as well. Hence, $y\in C^*(B_{n+1},\succsim)$.
    \item[ii.] $x^*\succsim y$: since $x^*\in C^*(B_n,\succsim)\Rightarrow x^*\succsim x, \forall x\in B_n$, hence $x^*\succsim x, \forall x\in B_n\cup{y}\Rightarrow x^*\in C^*(B_{n+1},\succsim)$
\end{enumerate}

Notice that when $B$ is finite, a stronger condition of $\succsim$ being acyclic and complete is equilavent to an induced choice rule $C^*(B,\succsim)\neq \varnothing$: 
\begin{theorem}\label{theo_acyclic_choice}
    For a finite $B$, $\succsim$ is complete and \textbf{acyclic} $\Leftrightarrow C^*(B,\succsim)\neq \varnothing$
\end{theorem}
$\succsim$ is acyclic mean that: $b_1\succsim b_2,b_2\succsim b_3,\cdots, b_{n-1}\succsim b_n\Rightarrow b_n\not\succsim b_1$. An example of transitive but not \textit{acyclic} relations is indifference $\sim$: $a_1\sim a_2\sim \cdots\sim a_n\Rightarrow a_n\sim a_1$.
A brief proof of Theorem \ref{theo_acyclic_choice} is:
\begin{enumerate}
    \item[i.] acyclic $\succsim\Rightarrow C^*(B,\succsim)\neq\varnothing$: Suppose if $C^*(B,\succsim)=\varnothing$, for $b_1\in B$, $b_1\notin C^*(B,\succsim)\Rightarrow \exists b_2$ s.t. $b_2\succsim b_1$. Continue this process, we can generate a sequence of $\cdots\succsim b_2\succsim b_1$, since $B$ is finite, this sequence must end at $b_n$. If $\succsim$ is acyclic, $b_1\not\succsim b_n$, this gives $b_n\succ b_1$, which would mean $b_n$ must be in $C^*(B,\succsim)$, contradicting.
    \item[ii.] $C^*(B,\succsim)\neq\varnothing\Rightarrow$ acyclic $\succsim$: Suppose $\succsim$ is not acyclic, then there exists $b_1\succsim b_2\succsim \cdots\succsim b_n\succsim b_1$, then for set $B=\{b_1,b_2,\cdots,b_n\}$, $\nexists b^*$ s.t. $b^*\succsim b_i \forall b_i\in B$, i.e., $C^*(B,\succsim)=\varnothing$.
\end{enumerate}

Of course, we want to extend this to the situation where $B$ is infinite. However, in general, it is possible that $C^*(B,\succsim)$ is empty (if you set the most preferred side of $B$ open, it would be impossible to choose based on the preferences).

With induced choice correspondence $C^*(B,\succsim)$ defined and non-emptyness proved, we can then say:
\begin{theorem}\label{thm_rational_leadto_WARP}
    If $\succsim$ is a rational preference relation, then the choice structure generated by $\succsim$, $(\mathcal{B},C^*(\cdot,\succsim))$, satisfies WARP
\end{theorem}

We can prove this theorem quite easily: $\forall B,B'$ suppose we have $x,y\in B\cap B'$ and $x\in C^*(B,\succsim),y\in C^*(B',\succsim)$, then $x\succsim a, \forall a\in B$ and $y\succsim b,\forall b\in B'$. Naturally, we have $x\succsim y$ since $y\in B$. By rationality (transitivity) of $\succsim$, we have $x\succsim y\succsim b,\forall b\in B'$, which means $x\in C^*(B',\succsim)$. This is precisely the definition of WARP

\subsection*{$(\mathcal{B},C(\cdot))$ satisfies WARP $\Rightarrow$ Rational $\succsim$}
The proof of this direction is more subtle, and is NOT necessarily a yes. Again, we start from a auxilary definition:
\begin{definition}\label{def_rationalize_choice}
    For a choice structure $(\mathcal{B},C(\cdot))$, a rational preference relation $\succsim$ \textbf{rationalizes} $C(\cdot)$ relative to $\mathcal{B}$ if $C(B)=C^*(B,\succsim), \forall B\in\mathcal{B}$.
\end{definition}

In words, if for all budget sets $B\in\mathcal{B}$, the choices generated by a rational $\succsim$, is just the choice rule $C(\cdot)$, $C(\cdot)$ is rationalized by $\succsim$. This is, in a sense, constructing an explanation of decision making behavior with preferences.

We already proved that $C^*(B,\succsim)$ satisfies WARP, which means that if a rationalizing preference relation to exist, WARP must be satisfied. However, if WARP is satisfied, a rationalizing preference relation does \textbf{NOT} necessarily exist.\footnote{A simple example is: $X=\{x,y,z\},\mathcal{B}=\{\{x,y\},\{y,z\},\{x,z\}\}$. Since $\mathcal{B}$ contains 3 binary menus, the choice structure $C(\{x,y\})=\{x\},C(\{y,z\})=\{y\},C(\{x,z\})=\{z\}$ vacuously satisfy WARP. But, this choice structure cannot be rationalized since it contradicts transitivity.}
Intuitiviely, more budget sets $B\in\mathcal{B}$ would mean that, to satisfy WARP, choice behavior would be restricted more, and it is easier to be self-contradicting. Therefore, to pin down a rational preference relation to rationalize $C(\cdot)$ relative to $\mathcal{B}$, we need to put some \textbf{restrictions on $\mathcal{B}$}.

\begin{theorem}\label{theorem_rationalizing_exist}
    If $(\mathcal{B},C(\cdot))$ is a choice structure that:
    \begin{enumerate}
        \item[i.] WARP is satisfied
        \item[ii.] $\mathcal{B}$ includes \textbf{all} subsets of $X$ of \textbf{up to 3} elements 
    \end{enumerate}
    then there exists a rational preference relation $\succsim$ s.t. $C(B)=C^*(B,\succsim),\forall B\in\mathcal{B}$. And this rational $\succsim$ is the \textbf{only} preference relation that can rationalize
     $(\mathcal{B},C(\cdot))$\footnote{The existence of a rationalizing preference relation $\succsim$ brings many interesting properties, one of them is \textit{path-invariance}: $\forall B_1,B_2\in\mathcal{B}, B_1\cup B_2 \in\mathcal{B}\land C(B_1)\cup C(B_2)\in\mathcal{B}\Rightarrow C(B_1\cup B_2)=C(C(B_1)\cup C(B_2))$, meaning that the decision problem can safely be subdivided. 
     A proof is: for $x\in C(B_1\cup B_2)$ and $y\in C(B_1)\cup C(B_2)$, since $C(B_1)\cup C(B_2) \subset B_1\cup B_2$, thus $x\in C(B_1\cup B_2)\Rightarrow x\in C(C(B_1)\cup C(B_2))$; for $x\in C(C(B_1)\cup C(B_2))$ and $y\in B_1\cup B_2$, we have $x\succsim z,\forall z\in C(B_1),C(B_2)\Rightarrow x\succsim w,\forall w\in B_1\cup B_2\Rightarrow x\in C(B_1\cup B_2)$}.
\end{theorem}

Now let's prove it, by examing the natural candidate for a rationalizing preference relation: the \textbf{revealed preference relation $\succsim^*$}:
\begin{enumerate}
    \item[\textbf{Step 1}] Prove that $\succsim^*$ is rational
    \begin{enumerate}
        \item[-] Completeness: By (ii) of Def.\ref{def_rationalize_choice}, all binary subsets of $X$ are in $\mathcal{B}$. Hence, $\{x,y\}\in\mathcal{B}$. For this binary menu, $C(\{x,y\})$ must contain either $x$ or $y$, therefore, $x\succsim^*y$ or $y\succsim^* x$ or both. Completeness proved.
        \item[-] Transitivity: $\forall\{x,y,z\}\in \mathcal{B}$, $C(\{x,y,z\})\neq \varnothing$. Suppose $x\succsim^*y,y\succsim^*z$, which implies that $x\in C(\{x,y\}),y\in C(\{y,z\})$, we then have three cases for $C(\{x,y,z\})$:
        \begin{enumerate}
            \item[a.] $x\in C(\{x,y,z\})$, WARP gives that $x\in C(\{x,z\})\Rightarrow x\succsim^*z$
            \item[b.] $y\in C(\{x,y,z\})$, we have $x\in C(\{x,y\})$. WARP gives $x\in C(\{x,y,z\})$ $\Rightarrow x\succsim^* z$
            \item[c.] $z\in C(\{x,y,z\})$, we have $y\in C(\{y,z\})$. WARP gives $y\in C(\{x,y,z\})$, and $x\in C(\{x,y\})$, WARP gives $x\in C(\{x,y,z\})\Rightarrow x\succsim^*z$
        \end{enumerate} 
        Hence, $x\succsim^*y,y\succsim^*z\Rightarrow x\succsim^* z$
    \end{enumerate} 
    \item[\textbf{Step 2}] Prove that $\succsim^*$ rationalizes $C(\cdot)$ on $\mathcal{B}$
    
    Now, we need to show $\forall B\in\mathcal{B}, C(B)=C^*(B,\succsim^*)$. Logically, this means the revealed preference $\succsim^*$ inferred from $C(\cdot)$ actually generates $C(\cdot)$. Formally, we prove it in 2 steps:
    \begin{enumerate}
        \item[a.] Suppose $x\in C(B)$, which means that $\forall y\in B, x\succsim^* y$ (by Def.\ref{def_revealed_pref}), hence $x\in C^*(B,\succsim^*)$ (by Def.\ref{def_induced_choice}). This proves $C(B)\subseteq C^*(B,\succsim^*)$
        \item[b.] Suppose $x\in C^*(B,\succsim^*)$, which means that $\forall y\in B, x\succsim^* y$ (by Def.\ref{def_induced_choice}). Therefore, $\forall y\in B$, there must exist a set $B_y\in\mathcal{B}$ s.t.
         $x,y\in B_y\Rightarrow x\in C(B_y)$. Since $C(B)\neq \varnothing$, suppose $z\in C(B)$, since $x\in C(B_z)$, WARP implies that $x\in C(B)$. This proves $C^*(B,\succsim^*)\subseteq C(B)$
    \end{enumerate}
    Together, we have $C(B)=C^*(B,\succsim^*)$.
    \item[\textbf{Step 3}] Prove $\succsim^*$ is the unique choice
    
    Since $\mathcal{B}$ includes all two-element subsets of $X$, the choice behavior in $C(\cdot)$ completely determines the pairwise preference relations over $X$ of any rationalizing preference.
\end{enumerate}

Now, it is \textbf{proved}! Notice that the main assumption(restriction) here is \textbf{$\mathcal{B}$ includes all subsets of $X$ of up to 3 elements}, this gives completeness, which is fundamental.

\subsection*{Things to keep in mind}
We have proved the twoway links of preferences and choices:
\begin{enumerate}
    \item[-] Rational $\succsim\Rightarrow (\mathcal{B},C^*(\cdot,\succsim))$ satisfies WARP (see Thm.\ref{thm_rational_leadto_WARP})
    \item[-] A WARP-satisfying, up-to-3-element $(\mathcal{B},C(\cdot))$ can be uniquely rationalized by a rational $\succsim$ (see Thm.\ref{theorem_rationalizing_exist})
\end{enumerate}
However, there are still something to keep in mind.

First, for a given choice structure $(\mathcal{B},C(\cdot))$, there may be \textbf{more than one} rationalizing preference relation $\succsim$ in general. Here is the simplest example:
For $X=\{x,y\},\mathcal{B}\{\{x\},\{y\}\}$ and the choice structure $C(\{x\})=\{x\}, C(\{y\}=\{y\})$. In this case, \textbf{ANY} relation preference relation of $X$ can rationalize $C(\cdot)$
This is related to both Def.\ref{def_rationalize_choice} and (ii) of Thm.\ref{theorem_rationalizing_exist}. Thm.\ref{theorem_rationalizing_exist} gives that if $\mathcal{B}$ contains \textbf{ALL binary} menus of $X$,
then there could be at most one rationalizing preference relation.

Second, the restriction for WARP$\Rightarrow$ rational $\succsim$, namely $\mathcal{B}$ containing all subsets of up to 3 elements, is too strong. For many economic problems, we will not consider all possible subsets, or limit ourselves to up-to-3-element ones. A strengthened version of WARP will be introduced later for that purpose.

Finally, up till now, we define a rationalizing preference as one: $C(B)=C^*(B,\succsim)$ (Def.\ref{def_rationalize_choice}). A common alternative would be to require only $C(B)\subset C^*(B,\succsim)$: if $C(B)$ is a \textbf{subset} of the most preferred choices generated by $\succsim$, i.e., $C^*(B,\succsim)$. This will allow indifferences
to be more than the situation of anything might be picked. And it is empirically intuitive in a sense that observed choices will never fully reveal decision makers' entire preferencing maximizing choice set. Naturally, $C(B)\subset C^*(B,\succsim)$ is weaker than $C(B)=C^*(B,\succsim)$. But $C(B)\subset C^*(B,\succsim)$ has an interesting property: the all-indifferent preference
will be able to rationalize \textit{any} choice behavior. Therefore, when $C(B)\subset C^*(B,\succsim)$ is used, you would always need to put some additional restrictions on the rationalizing preference relation for the specific economic context.

\section{Introducing Utility}\label{chap2:sec4}
%Now, with preferences and choices defined, and the linkage between the two established, we need to transfer these concepts into math for analytic studies.
This is exactly why utility functions are introduced: to assign a number and rank the elements in $X$ according to preferences.

\begin{definition}\label{def_utility}
    A function $u:X\rightarrow\mathbb{R}$ is a \textit{utility function representing relation} $\succsim$ if $\forall x,y\in X,x\succsim y\Leftrightarrow u(x)\geq u(y)$
\end{definition}

Notice that a utility function representing a preference relation $\succsim$ is NOT unique. \textbf{Rank-preserving} is the only requirement, hence, any 
strictly increasing function $f:\mathbb{R}\rightarrow\mathbb{R},v(x)=f(u(x))$ will also represent $\succsim$ as $u(\cdot)$. The logic is quite straight forward: for $x,y\in X$ and $u(\cdot)$ represents $\succsim$, then $x\succsim y \Leftrightarrow u(x)\geq u(y)$, for a strictly increasing $f(\cdot)$, $u(x)\geq u(y)\Leftrightarrow f(u(x))\geq f(u(y))\Leftrightarrow v(x)\geq v(y)$, hence $v(\cdot)$ represents $\succsim$ as well.
The major requirement here is \textbf{strictly increasing $f(\cdot)$}.

Two concepts to keep in mind:
\begin{enumerate}
    \item \textbf{Ordinal} properties of utility functions: the \textbf{invariant} properties of $u(\cdot)$ across all of its strictly increasing transformations $f(u(\cdot))$. Ranking (i.e. the preference represented by utility functions) is ordinal.
    \item \textbf{Cardinal} properties of utility functions: the \textbf{variant} properties of $u(\cdot)$ across all of its strictly increasing transformation $f(u(\cdot))$. Numerical values associated with the alternatives in $X$ (i.e. the magnitude of the differences between alternatives) is cardinal.
\end{enumerate}

The central theorem of utility functions is closely linked to rationality:
\begin{theorem}\label{thm_utility_is_rational}
    A preference relation $\succsim$ can be represented by a utility function $\Rightarrow$ $\succsim$ is rational
\end{theorem}

The proof is
\begin{enumerate}
    \item[-] \textit{Completeness}. Since $u(\cdot)$ represents preference relations between alternatives, and $u:X\rightarrow\mathbb{R}$, thus $\forall x,y\in X$, either $u(x)\geq u(y)$ or $u(y) \geq u(x)$. By Def.\ref{def_utility}, we have either $x\succsim y$ or $y\succsim x$, hence $\succsim$ is complete.
    \item[-] \textit{Transitivity}. For $x\succsim y,y\succsim z$. By Def.\ref{def_utility}, $u(x)\geq u(y),u(y)\geq u(z)$, hence $u(x)\geq u(z)\Rightarrow x\succsim z$.
\end{enumerate}

What about the other way? It is true, subject to some prerequisites:
\begin{theorem}\label{thm_rational_has_utility}
    $\succsim$ is rational and $X$ is \textbf{finite} $\Rightarrow$ there is a utility function representing $\succsim$.
\end{theorem}

The major prerequisite here is $X$ being \textbf{finite}. The proof is done by induction: Suppose there are $N$ elements in $X$:
\begin{enumerate}
    \item[-] When $N=1$, any number could be assigned to that element as its utility.
    \item[-] Suppose a rational $\succsim$ on $X={x_1,x_2,\cdots,x_{N-1}}$ could be represented by a utility function $u(\cdot)$. Without losing generality, we can assume $u(x_1)\leq u(x_2)\leq\cdots\leq u(x_{N-1})$. For the $N$th element $x_N$, by the rationality of $\succsim$, we have three scenarios:
    \begin{enumerate}
        \item[i] $\forall i\in {1,\cdots,N-1}, x_N\succsim x_i$: by Def.\ref{def_utility}, $u(x_N)\geq u(x_i)$.
        \item[ii] $\forall i\in {1,\cdots,N-1}, x_i\succsim x_N$:, $u(x_N)\leq u(x_i)$.
        \item[iii] $\exists i,j\in {1,\cdots,N-1}, i\neq j, x_j \succsim x_N\succsim x_i$: $u(x_j)\geq u(x_N)\geq u(x_i)$. By completeness and transitivity, ${x_1,x_2,\cdots,x_{N-1}}$ can be "divided" by $x_N$, meaning that for $I=\{i:x_N\succsim x_i\}$ and $J=\{j:s_j\succsim x_N\}$, $I\cup J=\{1,\cdots,N-1\}$. Note that we have assumed the index as the ranking, hence let $i^*=\max I,j^*=\min J$, $i^*+1=j^*$, hence we must have $u(x_i)\leq u(x_{i^*})\leq u(x_N)\leq u(x_{j^*})\leq u(x_j)$
    \end{enumerate}
    In all 3 scenarios, $u(\cdot)$ represents $\succsim$ on $X=\{x_1,\cdots,x_{N-1},x_N\}$ as well.
\end{enumerate}
With this induction, we prove Thm.\ref{thm_rational_has_utility}\footnote{Another way of proof is: Start with $x^{start}\in X$, define $W_{x}=\{y:y\prec x^{start}\}$ then $W_x$ is either empty or not: If not empty, pick $\tilde{x}\in W_{x}$, shrink $W_{x}$ to $\{y:y\prec \tilde{x}\}$ and repeat this procedure till $a\sim x^{stop}$ where $u(a)=0$, then $x^{stop}$ is the “lower bound” of the set. With this process, we can generate a utility function for any finite set $X$ that is rational.}.

\section{Commentary}\label{chap2:sec5}
%In this section, I discuss some of common commentaries on the standard preference model presented above.

\subsection{Preference model as a descriptive model}
A common complaint about the standard utility maximization/preference ranking model is that no one in reality actually calculates a number as utility before making choices.
This comment has a lot of sense to it since we rarely care about utility, let alone doing some math, before grocery shopping. But this observation does NOT invalidate the 
usefulness of preference/utility model.

The standard model does NOT regulate agents to consciously maximize utility, instead, it assumes individuals act \textit{as if} they maximize utility. Mathematically, we have already
proven that if choice behavior satisfies finite nonemptiness and WARP, then something will be chosen, and agents' choice behavior is just \textit{as if} it were preference driven, or the choice
behavior can be linked to a preference. If the set of choices is countable, then the preference-driven choice can be indexed by numbers, hence, becomes a mathematical question.

Utility/preference/choice system is considered as a description of choice behavior. Long as people do make a choice, and that choice satisfies WARP, we can always find a numerical way to 
\textit{describe} the behavioral pattern.

\subsection{Empirical limits}
To verify utility maximization as a model of choices over the choice space $X$, we need to check every subset $A$ of it. And we also need to know all of $C(A)$. Of course we have already managed to verify the two-way link between
preference, utility and choice for all choice menus with no more than 3 elements, but above that, it would be extremely difficult. Empirically, we will observe (at best) $C(A)$ for finitely many subsets of $X$, we would most likely
observe only the \textit{one} element that is selected out of $C(A)$ while failing to identify equally-preferred alternatives simply because they are not observed to be chosen. In these scenarios, how can we tell whether our observations
are aligned with utility maximization? In later chapters, we will come back to this problem.

\subsection{Framing}


\vspace{0.5cm}
\noindent\rule{\textwidth}{0.4pt}

For the content of this chapter, my main reference is Chapter 1 of \citet{mas1995microeconomic}. Section 1, Chapter 2 of \citet{kreps1990acourse} covers similar content but starts from strict preference $\succ$, it is a very
good complement to \citet{mas1995microeconomic}. Chapter 1 of \citet{kreps2013microeconomic} explores choice and preferences on infinite sets. Lecture 1 and 2 of \citet{ariel2012lecture} give a well organized, lecture-structured summary of
these contents, it is a very good read.

\chapter{Lagrange Maximization and Duality}
\input{chapter3/chapter3.tex}

\chapter{Stochastic Choice}

\chapter{Monotone Comparative Statics}

\chapter{Expected Utility and Decisionmaking under Uncertainty}

\chapter{Aggregation and the Existence of a Representative Consumer}

\chapter{Producer Theory}

\part{Game Theory}

\chapter{Nash Equilibrium and Bayesian Nash Equilibrium}

\chapter{Rationalizability and DOminant Strategies}

\chapter{Correlated Equilibrium}

\chapter{Dynamic Games and Refinements}

\chapter{Repeated Games/Folk Theorem}

\chapter{Recursive Methods in Repeated Games}

\part{Mechanism Design and Contract Theory}

\chapter{Arrow's Theorem and Social Choice}

\chapter{Boundaries of the Firm and Coase's Theorem}

\chapter{Implementation Concepts}

\chapter{The Revelation Principle}

\chapter{Auctions and Optimal Auctions}

\chapter{Efficient Implementation}

\chapter{Moral Hazard}

\chapter{Full Implementation}

\newpage
\bibliographystyle{plainnat}
\bibliography{ref.bib}

\end{document}