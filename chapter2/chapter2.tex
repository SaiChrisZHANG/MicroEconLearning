\minitoc

\vspace{0.5cm}
The second chapter focuses on the most fundamental decision unit of microeconomic theory: \textit{consumer}. The main reference is Chapter 2 and 3 of \citet{mas1995microeconomic}.

The basic setting of consumer demand study is \textit{market economy}, where the goods and services that the consumer may acquire and consume are available for purchase at known prices (or trade for other goods at know exchange rates).

In this chapter, we will focus on 2 major aspects of the consumer theory: choice and demand.
\begin{center}
    \begin{tabular}{rl}
    \hline
    \textbf{choice} & individual decision making analysis based on choice\\ 
    \textbf{demand} & individual decision making analysis based on preference \\ 
    \hline
    \end{tabular}
\end{center}

The starting point of individual decision problem is a \textit{set of possible (mutually exclusive) alternatives} from which the individual must choose. To model decision making process
on this set of alternatives, one can:
\begin{enumerate}
    \item[-] either start from the tastes, i.e., \textit{preference relations} of individuals, and set up the patterns of decision making with preferences
    \item[-] or, start from the actual actions of individuals, i.e. \textit{choices}, to deduct a pattern of decision making
\end{enumerate}

The two aspects of consumer theory are actually closely related to each other. Just like choices and preferences in Chapter 1, they are two sides of the same coin. However, they are NOT equivalent.
The major conclusion of choice-based consumer theory is WARP is essentially equivalent to the \textit{compensated law of demand}, but WARP imposes fewer restrictions on demand than preference-based theory,
hence, does NOT necessarily guarantee the existence of a rationalizing preference relation for consumer demand, therefore, \textit{strong axiom of revealed preference} is introduced.

\section{Basic Setting}\label{chap2:sec1}
First, we introduce the basic settings of a consumer's problem in a market economy. These concepts will keep reoccuring in the following sections.

\subsection{Commodities}\label{chap2:sec1:ssec1}
First, we need to define the goods and services the consumers consume. We do not actually care about what they specifically are, instead, we use a very abstract concept \textit{commodities}
to summarize and analyze them.
\begin{definition}{commodity bundle}{}
    Assume there are $L<\infty$ different commodities, a \textit{commodity vector} or \textit{commodity bundle} is a list of amounts of the different commodities:$$x=\left[x_1,\cdots,x_L\right]^T$$
\end{definition}
$x$ can be view as a point in an $\mathbb{R}^L$ space, i.e., the commodity space. Each entry $x_l$ of $x$ ($l=1,\cdots,L$) represents the amount of commodity $l$ consumed, hence, the vector is referred to
as \textit{consumption vector} or \textit{consumption bundle}.

Three things to keep in mind:
\begin{enumerate}
    \item[-] Time can be incorporated into this setting, namely, today's commodity is distinct from tomorrow's commodity, even if they are otherwise the same. The value of time will come back in later chapters and is crucial in a large strand of behavioral economic literature. Same logic applies to other limitations that are easily neglected, like geographic ones.
    \item[-] Negative entries can exists in a commodity vector, indicating debits or net outflows of goods. In a producing problem or exchange problem, negative entries of commodity vectors are not rare.
    \item[-] Consumption is quite flexible and comes in many format empirically, for the sake of data collection conveniency, consumption data are often aggregated monthly, quarterly for even annually. Meanwhle, some consumptions in the commodity vectors may not actually occur in the market.
\end{enumerate}

\subsection{The Consumption Set}\label{chap2:sec1:ssec2}
Consumptions are limited by a number of constraints, which will form a subset of commodity space $X\subset \mathcal{R}^L$. With in this subset, all possible commodity bundles can be 
consumed, this is exactly the definition of consumption sets (see \citet[Page 19-20]{mas1995microeconomic} for some simple examples of consumption sets).

For now, we will focus on the simplest consumption set: all possible non-negative commodity bundles:
$$X=\mathbb{R}^L_+=\{x\in \mathbb{R}^L: x_l\geq 0,\forall l=1,\cdots, L\}$$

It is easy to show that
$$\mathbb{R}^L_+ \text{ is a }\textbf{convex} \text{ set}$$

A brief proof: $\forall \vec{x},\vec{y}\in \mathbb{R}^L_+$ and $\forall\alpha \in [0,1]$, $\alpha\vec{x}+(1-\alpha)\vec{y}=[\alpha x_1+(1-\alpha)y_1,\cdots,$ $\alpha x_L+(1-\alpha)y_L]^T$. Since $x_i\geq 0,y_i\geq 0$, $\alpha x_i+(1-\alpha)y_i\geq 0 \Rightarrow\alpha\vec{x}+(1-\alpha)\vec{y}\in \mathbb{R}^L_+$.

Convexity of consumption sets is an essential assumption here, but some of the results do survive without the assumption fo convexity.

Although consuption sets are formed due to some constraints, but these constraints have nothing to do with consumers' budget (exogenous constraints). It is intuitive that with a large enough budget (infinitely large if you may), you can always afford any consumption bundle in a give consumption set. But what if consumers, as in reality, do have a budget constraint and cannot afford every bundle in the consumption set?

\subsection{Prices and Consumption Cost}\label{chap2:sec1:ssec3}
Budget constraints are an important economic constraint faced by consumers: one can only consume the commodity bundles that she can afford.

To formalize this constraint, we need to introduce the \textit{price vector}: 
$$\vec{p}=\left[p_1,\cdots,p_L\right]\in \mathbb{R}^L$$
This price vector contains unit price information for each of the $L$ commodities. They are all traded in the market and the price information of them is publicly quoted (the \textit{principle of completeness of markets}).
For simplicity, we assume $\vec{p}\gg 0$ i.e. $\forall l,p_l>0$\footnote{Of course, price can be negative, meaning that consumers are actually paid to consume the "bad "commodity, such as polution.}.

Another important assumption is the \textit{price-taking assumption}: consumers do NOT have the power to influence the prices. Here, each consumer only buys a small (neglectable) fraction of the total demand for commodities.

With prices defined, we can finally define the \textbf{economic-affordability constraint} of consumers: For a consumer with wealth $w$, a consumption bundle $\vec{x}\in \mathbb{R}^L_+$ is affordable if its total cost does NOT exceed the consumer's wealth level $w$, formally, $$\vec{p}\cdot \vec{x}=p_1x_1+p_2x_2+\cdots+p_Lx_L\leq w$$
With the two core assumptions stated above, consumers face a \textit{linear price schedule}.

\section{Walrasian Budgets}\label{chap2:sec2}
\subsection{Walrasian Budgets}
We have already defined the economic-affordability constraint of consumers, if we also limit consumption bundle $x$ to be non-negative, we would have the Walrasian, or competitive budget:

\begin{definition}
    The Walrasian, or competitive budget set $$B_{\mathbf{p},w}=\{x\in\mathbb{R}^L_+:\mathbf{p}\cdot\mathbf{x}\leq w\}$$ is the set of all feasible consumption bundles give market prices $\mathbf{p}$ and wealth $w$.
\end{definition}

From a Walrasian budget's point of view, a consumer can only choose a consumption bundle $x$ from $B_{\mathbf{p},w}$. An underlining assumption here is $w>0$, otherwise consumers cannot afford anything. We can also
separately define the "edge" of a Walrasian budget set as:
\begin{definition}
    The \textit{budget hyperplane} is the set $\{x\in\mathbb{R}^L_+:\mathbf{p}\cdot\mathbf{x}=w\}$
\end{definition}
It determines the upper bound of the budget set: with prices of other commodities ($\mathbf{p}_{-i}$) and wealth level $w$ fixed, the change of commodity $i$'s price $p_i$ will enlarge/shrink the budget set by moving the budget hyperplane.
Geometrically, the price vector $\mathbf{p}$ must be orthogonal to the budget hyperplane, we can think it this way: for any two bundles $\mathbf{x}$ and $\mathbf{x}'$ one the budget hyperplane, we must have $\mathbf{p}\cdot\mathbf{x}=\mathbf{p}\cdot\mathbf{x}'=w$, hence
$\mathbf{p}\cdot\Delta\mathbf{x}=\mathbf{p}\cdot(\mathbf{x}'-\mathbf{x})=0$ is always true.

A core feature of the Walrasian budget set is that it is \textbf{convex}: $\forall \mathbf{x},\mathbf{y}\in B_{\mathbf{p},w}, \alpha\in[0,1], \alpha\mathbf{x}+(1-\alpha)\mathbf{y}\in B_{\mathbf{p},w}$. This is very easy to prove: $\mathbf{x}\in\mathbb{R}^L_+\land\mathbf{y}\in\mathbb{R}^L_+\Rightarrow \alpha\mathbf{x}+(1-\alpha)\mathbf{y}\in \mathbb{R}^L_+$, $\mathbf{p}\cdot\mathbf{x}\leq w\land\mathbf{p}\cdot \mathbf{y}\leq w\Rightarrow \alpha(\mathbf{p}\cdot\mathbf{x})+(1-\alpha)(\mathbf{p}\cdot\mathbf{y})\leq w$.
Notice that the Walrasian budget set is not automatically convex. Its convexity is induced from the convexity of its superset (the consumption set), in this case $\mathbb{R}^L_+$. In general, it is easy to show that the Walrasian budget set will convex as long as its corresponding consumption set is convex.

Of course, it is perfectly possible that a consumer's budget is NOT convex (and not Walrasian, in that sense), the brilliant work of \citet{deaton1980economics} has documented and discussed many complicated consumption sets that are not convex.

\subsection{Walrasian Demand Function}
With Walrasian budgets defined, we can define \textit{Walrasian demand correspondence} and \textit{Walrasian demand function} as:
\begin{definition}
    For each \textit{price-wealth} pair $(\mathbf{p},w)$, Walrasian demand correspondence is the set of chosen consumption bundles, written as $\mathbf{x}(\mathbf{p},w)$. When $\mathbf{x}(\mathbf{p},w)$ is single-valued, it will be referred to as a \textbf{Walrasian demand function}.
\end{definition}

The two main assumptions of $\mathbf{x}(\mathbf{p},w)$ are:
\begin{definition}
    For a Walrasian demand correspondence $\mathbf{x}(\mathbf{p},w)$, we assume it is:
    \begin{enumerate}
        \item[1.] \textbf{Homogeneous of degree zero}: $\forall \mathbf{p},w$ and $\alpha >0$, $\mathbf{x}(\alpha \mathbf{p},\alpha w)=\mathbf{x}(\mathbf{p},w)$. This means that if wealth and prices change in the same proportion at the same time, consumer would not change her choice. 
        
        There are two implications of homogeneous of degree zero assumption:
        \begin{enumerate}
            \item[-] \textbf{Dimension reduction}: With this assumption, we can reduce $\mathbf{x}(\mathbf{p},w)$ to $\mathbf{x}(\tilde{\mathbf{p}},1)$ where $\tilde{\mathbf{p}}=\frac{1}{w}\mathbf{p}$, hence to a $L$-argument problem.
            \item[-] \textbf{Choice structure}: By homogeneity of degree zero, $\mathbf{x}(\mathbf{p},w)$ depends only on $\mathbf{p}$ and $w$, i.e. the budget set, then for the family of Walrasian budget sets $\mathcal{B}^*=\{B_{\mathbf{p},w}:\mathbf{p}\gg 0,w>0\}$, $(\mathcal{B}^*,\mathbf{x}(\cdot))$ is a choice structure. This choice structure does NOT include all possible subsets of $X$\sidenotes{$\leftarrow X$ is the set of all possible bundles}, in particular, not all two- and three-element subsets of $X$\footnote{This relates to the argument that when one consumption bundle $\mathbf{x^*}$ is infeasible, it not being chosen does NOT mean it is less preferred. This will be dealt with more carefully with more assumptions of preference-based demand.}, therefore, requires more assumptions to have a rationalizing preference.
        \end{enumerate}
             
        \item[2.] \textbf{Walras' law}: $\forall \gg 0,w>0$, $\mathbf{p}\cdot\mathbf{x}=w$ for every $\mathbf{x}\in \mathbf{x}(\mathbf{p},w)$. Walras' law means that a consumer want to spend all her wealth for consumption, every consumption bundle hence will exhaust $w$. Walras' law implies that goods are continuous.
        
        Notice that this is implicitly intertemporal, meaning that Walras' law consider lifetime resource allocation.
    \end{enumerate}
\end{definition}

Next, we disucess several basic definitions induced from the Walrasian demand function. For a demand function 
$$\mathbf{x}(\mathbf{p},w)=\left[x_1 (\mathbf{p},w),x_2 (\mathbf{p},w),\cdots, x_L(\mathbf{p},w) \right]^T$$ we have the following two effects:

\begin{description}
    \item[Wealth effects] \underline{Fix $\mathbf{p}$, get a function of $x(w)$}.
    
    Take the partial derivative of demand function on wealth: $$D_w x(\mathbf{x},w)=\left[\frac{\partial x_1(\mathbf{p},w)}{\partial w}, \frac{\partial x_2(\mathbf{p},w)}{\partial w},\cdots, \frac{\partial x_L(\mathbf{p},w)}{\partial w} \right]^T$$, then $\frac{\partial x_l(\mathbf{p},w)}{\partial w}$ is the \textit{wealth effect} for the $l$th commodity.
    Wealth effects can classify commodities into two types:
    \begin{enumerate}
        \item[-] \textbf{normal}: $\partial x_l(\mathbf{p},w)/\partial w\geq 0$
        \item[-] \textbf{inferior}: $\partial x_l(\mathbf{p},w)/\partial w <0$
    \end{enumerate} 
    \item[Price effects] \underline{For commodity $l$, fix $\mathbf{p}_{-l}$ and $w$, get a function of $x(p_l)$}.
    
    Take the partial derivative of demand function on price vector:
    $$D_{\mathbf{p}}x(\mathbf{p},w)= \begin{bmatrix}
        \nabla_{\mathbf{p}}x_1(\mathbf{p},w)\\
        \vdots\\
        \nabla_{\mathbf{p}}x_L(\mathbf{p},w) \end{bmatrix} 
    =
    \begin{bmatrix}
        \frac{\partial x_1(\mathbf{p},w)}{\partial p_1} & \cdots & \frac{\partial x_1(\mathbf{p},w)}{\partial p_L}\\
        \vdots & \ddots & \vdots\\
        \frac{\partial x_L(\mathbf{p},w)}{\partial p_1} & \cdots &\frac{\partial x_L(\mathbf{p},w)}{\partial p_L}
        \end{bmatrix}$$
\end{description}

For most goods, the price effects would be negative: you would buy more if the price is lower. However, there \textit{Giffen} goods (often low quality) and \textit{Veblen} goods (often luxurious) that have positive price effects.

Regarding wealth effects $\partial x_l(\mathbf{p},w)/partial w$ and price effects $\partial x_l(\mathbf{p},w)/\partial p_k$, we have the following two theorems:

\begin{theorem}\label{thm_homo_waldemand_thm1}
    If the Walrasian demand function $\mathbf{x}(\mathbf{p},w)$ is \textbf{homogeneous of degree zero}, the $\forall \mathbf{p},w$:
    $$\sum^L_{k=1}\frac{\partial x_l(\mathbf{p},w)}{\partial p_k}p_k +\frac{\partial x_l(\mathbf{p},w)}{\partial w}w=0,\forall l=1,\cdots,L$$
\end{theorem}

This is easy to proof: homogeneity of degree zero gives $\mathbf{x}(\alpha \mathbf{p},\alpha w)-\mathbf{x}(\mathbf{p},w)=0$, differentiating with respect to $\alpha$, get:
$$D_{\alpha \mathbf{p}}\mathbf{x}(\alpha \mathbf{p},\alpha w)\cdot \mathbf{p} + D_{\alpha w} \mathbf{x}(\alpha \mathbf{p},\alpha w)w=0$$
this is true for any $\alpha$, if we take $\alpha=1$, we get $D_{\mathbf{p}}\mathbf{x}(\mathbf{p}, w)\cdot \mathbf{p} + D_{ w} \mathbf{x}( \mathbf{p}, w)w=\mathbf{0}$, which is just the matrix notation of Thm.\ref{thm_homo_waldemand_thm1}.

Intuitively, this implies the price (substitution) effects and wealth (income) effects induced by the price change of one commodity, when weighted by the prices of other commodities and wealth, will cancel out.

Another theorem is induced from Walras' law:
\begin{theorem}\label{thm_walraslaw_waldemand_thm2}
    If the Walrasian emand function $\mathbf{x}(\mathbf{p},w)$ satisfies \textbf{Walras' law}, then for all $\mathbf{p},w$:
    $$\sum^L_{l=1}p_l\frac{\partial x_l(\mathbf{p},w)}{\partial p_k}+x_k(\mathbf{p},w)=0,\forall k=1,\cdots,L$$
    and
    $$\sum^L_{l=1}p_l\frac{\partial x_l(\mathbf{p},w)}{\partial w}=1$$
\end{theorem}

The proof is also easy: Walras' law gives $\mathbf{p}\cdot \mathbf{x}(\mathbf{p},w)=w$, take derivatives with respect to $\mathbf{p}$, get $\mathbf{p}\cdot D_{\mathbf{p}}\mathbf{x}(\mathbf{p},w)+\mathbf{x}(\mathbf{p},w)^T=\mathbf{0}^T$; take derivatives with repsect to $w$, get $\mathbf{p}\cdot D_{w}(\mathbf{p},w)=1$. The intuition is: a change in prices do NOT change the total expenditure, and the total expenditure will change by the same amount with the change in wealth.

If we define elasticities as:
\begin{align*} 
    \epsilon_{lk}=\frac{\partial x_l(\mathbf{p},w)/x_l(\mathbf{p},w)}{\partial p_k /p_k} && \text{ \% change in demand for $l$ per \% change in the price of $k$} \\ 
    \epsilon_{lw}=\frac{\partial x_l(\mathbf{p},w)/x_l(\mathbf{p},w)}{\partial w/w} && \text{ \% change in demand for $l$ per \% change in wealth $w$}
\end{align*}

We can rewrite Thm.\ref{thm_homo_waldemand_thm1} as $\sum^L_{k=1}\epsilon_{lk}(\mathbf{p},w)+\epsilon_{lw}(\mathbf{p},w)=0$ for $l=1,\cdots,L$: this directly expresses that an equal \% change in all prices and wealth leads to no change in demand, i.e., the homogeneity of degree zero.
And we can rewrite the two equations in Thm.\ref{thm_walraslaw_waldemand_thm2} as $\sum^L_{l=1}\frac{p_lx_l(\mathbf{p},w)}{w}\epsilon_{lk}(\mathbf{p},w)+\frac{p_kx_k(\mathbf{p},w)}{w}=0$ for $k=1,\cdots,L$ and $\sum^L_{l=1}\frac{p_lx_l(\mathbf{p},w)}{w}\epsilon_{lw}(\mathbf{p},w)=1$.

\subsection{WARP and Law of Demand}
Since for the family of Walrasian budget sets $\mathcal{B}^*=\{B_{\mathbf{p},w}:\mathbf{p}\gg 0,w>0\}$, $(\mathcal{B}^*,\mathbf{p}(\cdot))$ is a choice structure, naturally, we would like to check when WARP holds for this choice structure.

\begin{definition}\label{def_walrasian_WARP}
    A Walrasian demand function $\mathbf{x}(\mathbf{p},w)$ satisfies WARP if any two price-wealth conditions $(\mathbf{p},w)$ and $(\mathbf{p}',w')$ satisfies:
    $$\mathbf{p}\cdot \mathbf{x}(\mathbf{p}',w')\leq w\text{ and }\mathbf{x}(\mathbf{p}',w')\neq \mathbf{x}(\mathbf{p},w)\Rightarrow \mathbf{p}'\cdot\mathbf{x}(\mathbf{p},w)>w'$$
\end{definition}

The intuition is quite straightforward: If $\mathbf{p}\cdot \mathbf{x}(\mathbf{p}',w')\leq w$ and $\mathbf{x}(\mathbf{p}',w')\neq \mathbf{x}(\mathbf{p},w)$, the consumer chooses $\mathbf{x}(\mathbf{p},w)$ even when $\mathbf{x}(\mathbf{p}',w')$ is affordable. Hence, $\mathbf{x}(\mathbf{p},w)$ is preferred over $\mathbf{x}(\mathbf{p}',w')$, which means that the only reason why she chooses $\mathbf{x}(\mathbf{p},w)$ instead of $\mathbf{x}(\mathbf{p}',w')$ is that she can not afford $\mathbf{x}(\mathbf{p},w)$ at $(\mathbf{p}',w')$, i.e. $\mathbf{p}'\mathbf{x}(\mathbf{p},w)>w'$. An easier way to understand WARP is that we \textbf{CANNOT} have both $\mathbf{p}'\cdot \mathbf{x}(\mathbf{p},w)\leq w'$ and $\mathbf{p}\cdot \mathbf{x}(\mathbf{p}',w')\leq w$, unless $\mathbf{x}(\mathbf{p}',w')=\mathbf{x}(\mathbf{p},w)$.

It is easy to show that this definition is a special case of Def.\ref{def_WARP}: here we consider single-valued $C(\cdot)$ (function), then Def.\ref{def_WARP} gives that $\forall B,B'$ and $x,y\in B\cap B'$, $x=C(B),y=C(B')\Rightarrow x\in C(B')\Rightarrow x=y$. Rewrite this in the context of Walrasian demand functions, we have: for any $(\mathbf{p},w),(\mathbf{p}',w')$, if $\mathbf{p}\cdot\mathbf{x}(\mathbf{p}',w')\leq w$ and $\mathbf{p}'\cdot \mathbf{x}(\mathbf{p},w)\leq w'$, then $\mathbf{x}(\mathbf{p},w)=\mathbf{x}(\mathbf{p'},w')$. This is the contrapositive statement of Def.\ref{def_walrasian_WARP}. Hence the two definitions are equivalent.

\subsubsection*{WARP and compensated price changes}
WARP can also be stated in terms of compensated price changes. At $(\mathbf{p},w)$, the consumer chooses $\mathbf{x}(\mathbf{p},w)$, if she still want to afford $\mathbf{x}(\mathbf{p},w)$ at a new price $\mathbf{p}'$, she would need to adjust her wealth to $w'=\mathbf{p}'\cdot \mathbf{x}(\mathbf{p},w)$, this gives \textit{Slutsky wealth compensation} $\Delta w=w'-w=\Delta \mathbf{p}\cdot\mathbf{x}(\mathbf{p},w)=(\mathbf{p}'-\mathbf{p})\mathbf{x}(\mathbf{p},w)$, $\Delta \mathbf{p}$ is referred to as \textit{compensated price changes}. With these concepts defined, WARP implies:
\begin{theorem}\label{thm_WARP_to_lawofdemand}
    For any $\Delta \mathbf{p}$ from initial situation $(\mathbf{p},w)$ to $(\mathbf{p}',w')=(\mathbf{p}',\mathbf{p}'\cdot(\mathbf{x},w))$, if WARP holds, we have
    $$(\mathbf{p}'-\mathbf{p})\cdot[\mathbf{x}(\mathbf{p}',w')-\mathbf{x}(\mathbf{p},w)]\leq 0$$
    with strict inequality unless $\mathbf{x}(\mathbf{p},w)=\mathbf{x}(\mathbf{p}',w')$.
\end{theorem}

Here is a proof: rewrite the left-side, get
$$(\mathbf{p}'-\mathbf{p})\cdot[\mathbf{x}(\mathbf{p}',w')-\mathbf{x}(\mathbf{p},w)]=\mathbf{p}'\cdot[\mathbf{x}(\mathbf{p}',w')-\mathbf{x}(\mathbf{p},w)]-\mathbf{p}\cdot[\mathbf{x}(\mathbf{p}',w')-\mathbf{x}(\mathbf{p},w)]$$
By Walras' law, $\mathbf{p}'\cdot\mathbf{x}(\mathbf{p}',w')=w'$, $\mathbf{p}\cdot \mathbf{x}(\mathbf{p},w)=w$, also by assumption of compensated price changes, $\mathbf{p}'\cdot\mathbf{x}(\mathbf{p},w)=w'$, if if WARP holds, since $\mathbf{x}(\mathbf{p},w)$ is affordable at $(\mathbf{p}',w')$, $\mathbf{x}(\mathbf{p}',w')$ must NOT be affordable at $(\mathbf{p},w)$, i.e., $\mathbf{p}\cdot \mathbf{x}(\mathbf{p}',w')>w$, therefore, we have:
$$(\mathbf{p}'-\mathbf{p})\cdot[\mathbf{x}(\mathbf{p}',w')-\mathbf{x}(\mathbf{p},w)]= \underbrace{\mathbf{p}'\cdot[\mathbf{x}(\mathbf{p}',w')-\mathbf{x}(\mathbf{p},w)]}_{=0} - \underbrace{\mathbf{p}\cdot[\mathbf{x}(\mathbf{p}',w')-\mathbf{x}(\mathbf{p},w)]}_{>0}<0$$

This Theorem goes both way, that is
\begin{theorem}\label{thm_lawofdemand_to_WARP}
    If for any compensated price changes, $(\mathbf{p}'-\mathbf{p})\cdot[\mathbf{x}(\mathbf{p}',w')-\mathbf{x}(\mathbf{p},w)]\leq 0$ holds, the WARP is satisfied.
\end{theorem}
We proof the contrapositive: if WARP is violated, there exists a compensated price change such that $(\mathbf{p}'-\mathbf{p})\cdot[\mathbf{x}(\mathbf{p}',w')-\mathbf{x}(\mathbf{p},w)]>0$. A violation of WARP gives that for $(\mathbf{p},w)$ and $(\mathbf{\mathbf{p}',w'})$ such that $\mathbf{x}(\mathbf{p},w)\neq \mathbf{x}(\mathbf{p}',w')$, $\mathbf{p}'\cdot \mathbf{x}(\mathbf{p},w)\leq w'$ and $\mathbf{p}\cdot \mathbf{x}(\mathbf{p}',w')\leq w$ can both be satisfied. The proof is done in 2 steps:
\begin{description}
    \item[Step 1] Prove the fact that: 
    
    If any two price-wealth pairs $(\mathbf{p},w),(\mathbf{p}',w')$, $\mathbf{p}\cdot\mathbf{x}(\mathbf{p}',w')=w,\mathbf{x}(\mathbf{p}',w')\neq \mathbf{x}(\mathbf{p},w)\Rightarrow \mathbf{p}'\cdot \mathbf{x}(\mathbf{p},w)>w'$, then WARP holds. 
    
    Again, we prove the contrapositive of this proposition. Let $(\mathbf{p}',w')$ and $(\mathbf{p}'',w'')$ violates WARP such that $\mathbf{x}(\mathbf{p}',w')\neq \mathbf{x}(\mathbf{p}'',w''),\mathbf{p}'\cdot\mathbf{x}(\mathbf{p}'',w'')\leq w'$ and $\mathbf{p}''\cdot\mathbf{x}(\mathbf{p}',w')\leq w''$. We have two scenarios:
    \begin{enumerate}
        \item[-] $\mathbf{p}'\cdot\mathbf{x}(\mathbf{p}'',w'')= w'$ or $\mathbf{p}''\cdot\mathbf{x}(\mathbf{p}',w')= w''$ or both: It is easy to show that the condition of the price-wealth pairs are violated.
        \item[-] $\mathbf{p}'\cdot\mathbf{x}(\mathbf{p}'',w'')> w'$ and $\mathbf{p}''\cdot\mathbf{x}(\mathbf{p}',w')>w''$: we can construct a price-wealth pair $(\mathbf{p},w)$ such that both $\mathbf{x}(\mathbf{p}',x')$ and $\mathbf{x}(\mathbf{p}'',x'')$ are affordable. By picking an $\alpha \in (0,1)$ to linearly combine $p'$ and $p''$, we can have:
        $$(\alpha \mathbf{p}'+(1-\alpha)\mathbf{p}'')\cdot \mathbf{x}(\mathbf{p}',w')= (\alpha \mathbf{p}'+(1-\alpha)\mathbf{p}'')\cdot \mathbf{x}(\mathbf{p}'',w'')$$\footnote{Why this construction works? Since $A = \mathbf{p}'\cdot \mathbf{x}(\mathbf{p}',w')= w'< \mathbf{p}'\cdot \mathbf{x}(\mathbf{p}'',w'')=A'$, $B=\mathbf{p}''\cdot \mathbf{x}(\mathbf{p}',w')>w''= \mathbf{p}''\cdot \mathbf{x}(\mathbf{p}'',w'')=B'$, therefore if $\alpha$ is properly chosen, we can achieve $\alpha A+(1-\alpha)B = \alpha A'+(1-\alpha) B'$ with $A<A',B>B'$.}
    \end{enumerate} 
\end{description}


\section{Linking Preferences with Choices}\label{chap2:sec3}
%Now we have two major approaches of decision making process: preference relations in Section \ref{chap1:sec1} and choice rules in Section \ref{chap1:sec2},
what we need to do is to link them. This linkage will emerge when we examine two central assumptions: \textbf{rationality} and \textbf{WARP}. So the major question here is: 
$$\textbf{rational}\succsim\xLeftrightarrow{???} (\mathcal{B},C(\cdot))\textbf{ satisfies WARP}$$
And the answer is: \textit{YES!} but not exactly. Now let's dig in.

\subsection*{Rational $\succsim\Rightarrow(\mathcal{B},C(\cdot))$ satisfies WARP}

First, $\textbf{rational}\succsim \Rightarrow (\mathcal{B},C(\cdot))\textbf{ satisfies WARP}$ is a big YES. To prove this, we need to define \textit{induced choice correspondence}:
\begin{definition}\label{def_induced_choice}
    Given a \textbf{rational} $\succsim$ on $X$, if the decision maker faces a nonempty subset of alternatives $B\subset X$, by maximizing her preference, she would choose any one of the elements in the 
    \textit{induced choice correspondence}: $C^*(B,\succsim)=\{x\in B:x\succsim y, \forall y\in B\}$
\end{definition}

The induced choice correspondence $C^*(B,\succsim)$ has an important property: 
\begin{theorem}
    if $X$ is finite, $C^*(B,\succsim)$ will be \textbf{nonempty}.
\end{theorem}

A brief proof of this proposition is: If $X$ is finite, $B$ is finite as well. We will prove by induction. Starting from $|B|=1$, the only element of $B$ is in $C^*(B,\succsim)$. Now suppose $C^*(B,\succsim)$ is nonempty when $|B_n|=n$, 
let $x^*\in C^*(B_n,\succsim)$; when $|B_{n+1}|=n+1$, let the $n+1$th element $y$ $(\{y\}=B_{n+1}\setminus B_n)$. By the completeness of a rational $\succsim$, either $y\succsim x^*$ or $x^*\succsim y$:
\begin{enumerate}
    \item[i.] $y\succsim x^*$: since $x^*\in C^*(B_n,\succsim)\Rightarrow x^*\succsim x, \forall x\in B_n$. By transitivity of $\succsim$, $y\succsim x,\forall\in B_n$. By completeness, $y\succsim y$ as well. Hence, $y\in C^*(B_{n+1},\succsim)$.
    \item[ii.] $x^*\succsim y$: since $x^*\in C^*(B_n,\succsim)\Rightarrow x^*\succsim x, \forall x\in B_n$, hence $x^*\succsim x, \forall x\in B_n\cup{y}\Rightarrow x^*\in C^*(B_{n+1},\succsim)$
\end{enumerate}

Notice that when $B$ is finite, a stronger condition of $\succsim$ being acyclic and complete is equilavent to an induced choice rule $C^*(B,\succsim)\neq \varnothing$: 
\begin{theorem}\label{theo_acyclic_choice}
    For a finite $B$, $\succsim$ is complete and \textbf{acyclic} $\Leftrightarrow C^*(B,\succsim)\neq \varnothing$
\end{theorem}
$\succsim$ is acyclic mean that: $b_1\succsim b_2,b_2\succsim b_3,\cdots, b_{n-1}\succsim b_n\Rightarrow b_n\not\succsim b_1$. An example of transitive but not \textit{acyclic} relations is indifference $\sim$: $a_1\sim a_2\sim \cdots\sim a_n\Rightarrow a_n\sim a_1$.
A brief proof of Theorem \ref{theo_acyclic_choice} is:
\begin{enumerate}
    \item[i.] acyclic $\succsim\Rightarrow C^*(B,\succsim)\neq\varnothing$: Suppose if $C^*(B,\succsim)=\varnothing$, for $b_1\in B$, $b_1\notin C^*(B,\succsim)\Rightarrow \exists b_2$ s.t. $b_2\succsim b_1$. Continue this process, we can generate a sequence of $\cdots\succsim b_2\succsim b_1$, since $B$ is finite, this sequence must end at $b_n$. If $\succsim$ is acyclic, $b_1\not\succsim b_n$, this gives $b_n\succ b_1$, which would mean $b_n$ must be in $C^*(B,\succsim)$, contradicting.
    \item[ii.] $C^*(B,\succsim)\neq\varnothing\Rightarrow$ acyclic $\succsim$: Suppose $\succsim$ is not acyclic, then there exists $b_1\succsim b_2\succsim \cdots\succsim b_n\succsim b_1$, then for set $B=\{b_1,b_2,\cdots,b_n\}$, $\nexists b^*$ s.t. $b^*\succsim b_i \forall b_i\in B$, i.e., $C^*(B,\succsim)=\varnothing$.
\end{enumerate}

With induced choice correspondence $C^*(B,\succsim)$ defined and non-emptyness proved, we can then say:
\begin{theorem}\label{thm_rational_leadto_WARP}
    If $\succsim$ is a rational preference relation, then the choice structure generated by $\succsim$, $(\mathcal{B},C^*(\cdot,\succsim))$, satisfies WARP
\end{theorem}

We can prove this theorem quite easily: $\forall B,B'$ suppose we have $x,y\in B\cap B'$ and $x\in C^*(B,\succsim),y\in C^*(B',\succsim)$, then $x\succsim a, \forall a\in B$ and $y\succsim b,\forall b\in B'$. Naturally, we have $x\succsim y$ since $y\in B$. By rationality (transitivity) of $\succsim$, we have $x\succsim y\succsim b,\forall b\in B'$, which means $x\in C^*(B',\succsim)$. This is precisely the definition of WARP

\subsection*{$(\mathcal{B},C(\cdot))$ satisfies WARP $\Rightarrow$ Rational $\succsim$}
The proof of this direction is more subtle, and is NOT necessarily a yes. Again, we start from a auxilary definition:
\begin{definition}\label{def_rationalize_choice}
    For a choice structure $(\mathcal{B},C(\cdot))$, a rational preference relation $\succsim$ \textbf{rationalizes} $C(\cdot)$ relative to $\mathcal{B}$ if $C(B)=C^*(B,\succsim), \forall B\in\mathcal{B}$.
\end{definition}

In words, if for all budget sets $B\in\mathcal{B}$, the choices generated by a rational $\succsim$, is just the choice rule $C(\cdot)$, $C(\cdot)$ is rationalized by $\succsim$. This is, in a sense, constructing an explanation of decision making behavior with preferences.

We already proved that $C^*(B,\succsim)$ satisfies WARP, which means that if a rationalizing preference relation to exist, WARP must be satisfied. However, if WARP is satisfied, a rationalizing preference relation does \textbf{NOT} necessarily exist.\footnote{A simple example is: $X=\{x,y,z\},\mathcal{B}=\{\{x,y\},\{y,z\},\{x,z\}\}$. Since $\mathcal{B}$ contains 3 binary menus, the choice structure $C(\{x,y\})=\{x\},C(\{y,z\})=\{y\},C(\{x,z\})=\{z\}$ vacuously satisfy WARP. But, this choice structure cannot be rationalized since it contradicts transitivity.}
Intuitiviely, more budget sets $B\in\mathcal{B}$ would mean that, to satisfy WARP, choice behavior would be restricted more, and it is easier to be self-contradicting. Therefore, to pin down a rational preference relation to rationalize $C(\cdot)$ relative to $\mathcal{B}$, we need to put some \textbf{restrictions on $\mathcal{B}$}.

\begin{theorem}\label{theorem_rationalizing_exist}
    If $(\mathcal{B},C(\cdot))$ is a choice structure that:
    \begin{enumerate}
        \item[i.] WARP is satisfied
        \item[ii.] $\mathcal{B}$ includes \textbf{all} subsets of $X$ of \textbf{up to 3} elements 
    \end{enumerate}
    then there exists a rational preference relation $\succsim$ s.t. $C(B)=C^*(B,\succsim),\forall B\in\mathcal{B}$. And this rational $\succsim$ is the \textbf{only} preference relation that can rationalize $(\mathcal{B},C(\cdot))$.
\end{theorem}

Now let's prove it, by examing the natural candidate for a rationalizing preference relation: the \textbf{revealed preference relation $\succsim^*$}:
\begin{enumerate}
    \item[\textbf{Step 1}] Prove that $\succsim^*$ is rational
    \begin{enumerate}
        \item[-] Completeness: By (ii) of Def.\ref{def_rationalize_choice}, all binary subsets of $X$ are in $\mathcal{B}$. Hence, $\{x,y\}\in\mathcal{B}$. For this binary menu, $C(\{x,y\})$ must contain either $x$ or $y$, therefore, $x\succsim^*y$ or $y\succsim^* x$ or both. Completeness proved.
        \item[-] Transitivity: $\forall\{x,y,z\}\in \mathcal{B}$, $C(\{x,y,z\})\neq \varnothing$. Suppose $x\succsim^*y,y\succsim^*z$, which implies that $x\in C(\{x,y\}),y\in C(\{y,z\})$, we then have three cases for $C(\{x,y,z\})$:
        \begin{enumerate}
            \item[a.] $x\in C(\{x,y,z\})$, WARP gives that $x\in C(\{x,z\})\Rightarrow x\succsim^*z$
            \item[b.] $y\in C(\{x,y,z\})$, we have $x\in C(\{x,y\})$. WARP gives $x\in C(\{x,y,z\})$ $\Rightarrow x\succsim^* z$
            \item[c.] $z\in C(\{x,y,z\})$, we have $y\in C(\{y,z\})$. WARP gives $y\in C(\{x,y,z\})$, and $x\in C(\{x,y\})$, WARP gives $x\in C(\{x,y,z\})\Rightarrow x\succsim^*z$
        \end{enumerate} 
        Hence, $x\succsim^*y,y\succsim^*z\Rightarrow x\succsim^* z$
    \end{enumerate} 
    \item[\textbf{Step 2}] Prove that $\succsim^*$ rationalizes $C(\cdot)$ on $\mathcal{B}$
    
    Now, we need to show $\forall B\in\mathcal{B}, C(B)=C^*(B,\succsim^*)$. Logically, this means the revealed preference $\succsim^*$ inferred from $C(\cdot)$ actually generates $C(\cdot)$. Formally, we prove it in 2 steps:
    \begin{enumerate}
        \item[a.] Suppose $x\in C(B)$, which means that $\forall y\in B, x\succsim^* y$ (by Def.\ref{def_revealed_pref}), hence $x\in C^*(B,\succsim^*)$ (by Def.\ref{def_induced_choice}). This proves $C(B)\subseteq C^*(B,\succsim^*)$
        \item[b.] Suppose $x\in C^*(B,\succsim^*)$, which means that $\forall y\in B, x\succsim^* y$ (by Def.\ref{def_induced_choice}). Therefore, $\forall y\in B$, there must exist a set $B_y\in\mathcal{B}$ s.t.
         $x,y\in B_y\Rightarrow x\in C(B_y)$. Since $C(B)\neq \varnothing$, suppose $z\in C(B)$, since $x\in C(B_z)$, WARP implies that $x\in C(B)$. This proves $C^*(B,\succsim^*)\subseteq C(B)$
    \end{enumerate}
    Together, we have $C(B)=C^*(B,\succsim^*)$.
    \item[\textbf{Step 3}] Prove $\succsim^*$ is the unique choice
    
    Since $\mathcal{B}$ includes all two-element subsets of $X$, the choice behavior in $C(\cdot)$ completely determines the pairwise preference relations over $X$ of any rationalizing preference.
\end{enumerate}

Now, it is \textbf{proved}! Notice that the main assumption(restriction) here is \textbf{$\mathcal{B}$ includes all subsets of $X$ of up to 3 elements}, this gives completeness, which is fundamental.

\subsection*{Two things to keep in mind}
We have proved the twoway links of preferences and choices:
\begin{enumerate}
    \item[-] Rational $\succsim\Rightarrow (\mathcal{B},C^*(\cdot,\succsim))$ satisfies WARP (see Thm.\ref{thm_rational_leadto_WARP})
    \item[-] A WARP-satisfying, up-to-3-element $(\mathcal{B},C(\cdot))$ can be uniquely rationalized by a rational $\succsim$ (see Thm.\ref{theorem_rationalizing_exist})
\end{enumerate}
However, there are still something to keep in mind.

First, for a given choice structure $(\mathcal{B},C(\cdot))$, there my be \textbf{more than one} rationalizing preference relation $\succsim$ in general. Here is the simplest example:
For $X=\{x,y\},\mathcal{B}\{\{x\},\{y\}\}$ and the choice structure $C(\{x\})=\{x\}, C(\{y\}=\{y\})$. In this case, \textbf{ANY} relation preference relation of $X$ can rationalize $C(\cdot)$
This is related to both Def.\ref{def_rationalize_choice} and (ii) of Thm.\ref{theorem_rationalizing_exist}. Thm.\ref{theorem_rationalizing_exist} gives that if $\mathcal{B}$ contains \textbf{ALL binary} menus of $X$,
then there could be at most one rationalizing preference relation.

Second, the restriction for WARP$\Rightarrow$ rational $\succsim$, namely $\mathcal{B}$ containing all subsets of up to 3 elements, is too strong. For many economic problems, we will not consider all possible subsets, or limit ourselves to up-to-3-element ones. A strengthened version of WARP will be introduced later for that purpose.

Finally, up till now, we define a rationalizing preference as one: $C(B)=C^*(B,\succsim)$ (Def.\ref{def_rationalize_choice}). A common alternative would be to require only $C(B)\subset C^*(B,\succsim)$: if $C(B)$ is a \textbf{subset} of the most preferred choices generated by $\succsim$, i.e., $C^*(B,\succsim)$. This will allow indifferences
to be more than the situation of anything might be picked. And it is empirically intuitive in a sense that observed choices will never fully reveal decision makers' entire preferencing maximizing choice set. Naturally, $C(B)\subset C^*(B,\succsim)$ is weaker than $C(B)=C^*(B,\succsim)$. But $C(B)\subset C^*(B,\succsim)$ has an interesting property: the all-indifferent preference
will be able to rationalize \textit{any} choice behavior. Therefore, when $C(B)\subset C^*(B,\succsim)$ is used, you would always need to put some additional restrictions on the rationalizing preference relation for the specific economic context.

\section{Introducing Utility}\label{chap2:sec4}
%Now, with preferences and choices defined, and the linkage between the two established, we need to transfer these concepts into math for analytic studies.
This is exactly why utility functions are introduced: to assign a number and rank the elements in $X$ according to preferences.

\begin{definition}
    A function $u:X\rightarrow\mathbb{R}$ is a \textit{utility function representing}
\end{definition}

\section{Commentary}\label{chap2:sec5}
%In this section, I discuss some of common commentaries on the standard preference model presented above.

\subsection{Preference model as a descriptive model}
A common complaint about the standard utility maximization/preference ranking model is that no one in reality actually calculates a number as utility before making choices.
This comment has a lot of sense to it since we rarely care about utility, let alone doing some math, before grocery shopping. But this observation does NOT invalidate the 
usefulness of preference/utility model.

The standard model does NOT regulate agents to consciously maximize utility, instead, it assumes individuals act \textit{as if} they maximize utility. Mathematically, we have already
proven that if choice behavior satisfies finite nonemptiness and WARP, then something will be chosen, and agents' choice behavior is just \textit{as if} it were preference driven, or the choice
behavior can be linked to a preference. If the set of choices is countable, then the preference-driven choice can be indexed by numbers, hence, becomes a mathematical question.

Utility/preference/choice system is considered as a description of choice behavior. Long as people do make a choice, and that choice satisfies WARP, we can always find a numerical way to 
\textit{describe} the behavioral pattern.

\subsection{Empirical limits}


\vspace{0.5cm}
\noindent\rule{\textwidth}{0.4pt}

For the content of this chapter, my main reference is Chapter 1 of \citet{mas1995microeconomic}. Section 1, Chapter 2 of \citet{kreps1990acourse} covers similar content but starts from strict preference $\succ$, it is a very
good complement to \citet{mas1995microeconomic}. Chapter 1 of \citet{kreps2013microeconomic} explores choice and preferences on infinite sets. Lecture 1 and 2 of \citet{ariel2012lecture} give a well organized, lecture-structured summary of
these contents, it is a very good read.