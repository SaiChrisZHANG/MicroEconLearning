\minitoc

\vspace{0.5cm}
The second chapter focuses on the most fundamental decision unit of microeconomic theory: \textit{consumer}. The main reference is Chapter 2 and 3 of \citet{mas1995microeconomic}.

The basic setting of consumer demand study is \textit{market economy}, where the goods and services that the consumer may acquire and consume are available for purchase at known prices (or trade for other goods at know exchange rates).

In this chapter, we will focus on 2 major aspects of the consumer theory: choice and demand.
\begin{center}
    \begin{tabular}{rl}
    \hline
    \textbf{choice} & individual decision making analysis based on choice\\ 
    \textbf{demand} & individual decision making analysis based on preference \\ 
    \hline
    \end{tabular}
\end{center}

The starting point of individual decision problem is a \textit{set of possible (mutually exclusive) alternatives} from which the individual must choose. To model decision making process
on this set of alternatives, one can:
\begin{enumerate}
    \item[-] either start from the tastes, i.e., \textit{preference relations} of individuals, and set up the patterns of decision making with preferences
    \item[-] or, start from the actual actions of individuals, i.e. \textit{choices}, to deduct a pattern of decision making
\end{enumerate}

The two aspects of consumer theory are actually closely related to each other. Just like choices and preferences in Chapter 1, they are two sides of the same coin. However, they are NOT equivalent.
The major conclusion of choice-based consumer theory is WARP is essentially equivalent to the \textit{compensated law of demand}, but WARP imposes fewer restrictions on demand than preference-based theory,
hence, does NOT necessarily guarantee the existence of a rationalizing preference relation for consumer demand, therefore, \textit{strong axiom of revealed preference} is introduced.

\section{Basic setting}\label{chap2:sec1}
%First, we introduce the basic settings of a consumer's problem in a market economy. These concepts will keep reoccuring in the following sections.

\subsection{Commodities}\label{chap2:sec1:ssec1}
First, we need to define the goods and services the consumers consume. We do not actually care about what they specifically are, instead, we use a very abstract concept \textit{commodities}
to summarize and analyze them.
\begin{definition}{commodity bundle}{}
    Assume there are $L<\infty$ different commodities, a \textit{commodity vector} or \textit{commodity bundle} is a list of amounts of the different commodities:$$x=\left[x_1,\cdots,x_L\right]^T$$
\end{definition}
$x$ can be view as a point in an $\mathbb{R}^L$ space, i.e., the commodity space. Each entry $x_l$ of $x$ ($l=1,\cdots,L$) represents the amount of commodity $l$ consumed, hence, the vector is referred to
as \textit{consumption vector} or \textit{consumption bundle}.

Three things to keep in mind:
\begin{enumerate}
    \item[-] Time can be incorporated into this setting, namely, today's commodity is distinct from tomorrow's commodity, even if they are otherwise the same. The value of time will come back in later chapters and is crucial in a large strand of behavioral economic literature. Same logic applies to other limitations that are easily neglected, like geographic ones.
    \item[-] Negative entries can exists in a commodity vector, indicating debits or net outflows of goods. In a producing problem or exchange problem, negative entries of commodity vectors are not rare.
    \item[-] Consumption is quite flexible and comes in many format empirically, for the sake of data collection conveniency, consumption data are often aggregated monthly, quarterly for even annually. Meanwhle, some consumptions in the commodity vectors may not actually occur in the market.
\end{enumerate}

\subsection{The Consumption Set}\label{chap2:sec1:ssec2}
Consumptions are limited by a number of constraints, which will form a subset of commodity space $X\subset \mathcal{R}^L$. With in this subset, all possible commodity bundles can be 
consumed, this is exactly the definition of consumption sets (see \citet[Page 19-20]{mas1995microeconomic} for some simple examples of consumption sets).

For now, we will focus on the simplest consumption set: all possible non-negative commodity bundles:
$$X=\mathbb{R}^L_+=\{x\in \mathbb{R}^L: x_l\geq 0,\forall l=1,\cdots, L\}$$

It is easy to show that
$$\mathbb{R}^L_+ \text{ is a }\textbf{convex} \text{ set}$$

A brief proof: $\forall \vec{x},\vec{y}\in \mathbb{R}^L_+$ and $\forall\alpha \in [0,1]$, $\alpha\vec{x}+(1-\alpha)\vec{y}=[\alpha x_1+(1-\alpha)y_1,\cdots,$ $\alpha x_L+(1-\alpha)y_L]^T$. Since $x_i\geq 0,y_i\geq 0$, $\alpha x_i+(1-\alpha)y_i\geq 0 \Rightarrow\alpha\vec{x}+(1-\alpha)\vec{y}\in \mathbb{R}^L_+$.

Convexity of consumption sets is an essential assumption here, but some of the results do survive without the assumption fo convexity.

Although consuption sets are formed due to some constraints, but these constraints have nothing to do with consumers' budget (exogenous constraints). It is intuitive that with a large enough budget (infinitely large if you may), you can always afford any consumption bundle in a give consumption set. But what if consumers, as in reality, do have a budget constraint and cannot afford every bundle in the consumption set?

\subsection{Prices and Consumption Cost}\label{chap2:sec1:ssec3}
Budget constraints are an important economic constraint faced by consumers: one can only consume the commodity bundles that she can afford.

To formalize this constraint, we need to introduce the \textit{price vector}: 
$$\vec{p}=\left[p_1,\cdots,p_L\right]\in \mathbb{R}^L$$
This price vector contains unit price information for each of the $L$ commodities. They are all traded in the market and the price information of them is publicly quoted (the \textit{principle of completeness of markets}).
For simplicity, we assume $\vec{p}\gg 0$ i.e. $\forall l,p_l>0$\footnote{Of course, price can be negative, meaning that consumers are actually paid to consume the "bad "commodity, such as polution.}.

Another important assumption is the \textit{price-taking assumption}: consumers do NOT have the power to influence the prices. Here, each consumer only buys a small (neglectable) fraction of the total demand for commodities.

With prices defined, we can finally define the \textbf{economic-affordability constraint} of consumers: For a consumer with wealth $w$, a consumption bundle $\vec{x}\in \mathbb{R}^L_+$ is affordable if its total cost does NOT exceed the consumer's wealth level $w$, formally, $$\vec{p}\cdot \vec{x}=p_1x_1+p_2x_2+\cdots+p_Lx_L\leq w$$
With the two core assumptions stated above, consumers face a \textit{linear price schedule}.

\section{Choice Rules}\label{chap2:sec2}
%Next, we approach the theory of decision making from choice behavior itself. Formally, choice behavior is represented by means of a \textit{choice structure} $(\mathcal{B},C(\cdot))$. Now, we define choice structure $(\mathcal{B},C(\cdot))$:
\begin{definition}
    A choice structure $(\mathcal{B},C(\cdot))$ has two ingredients:
    \begin{enumerate}
        \item[-] $\mathcal{B}\subset \mathcal{P}(X)\setminus\varnothing$, where $\mathcal{P}(X)$ is the power set of $X$. This means, every element $B\in \mathcal{B}$ is a subset of $X$\footnote{The elements $B\in\mathcal{B}$ are so-called \textit{budget sets}. The budget sets in $\mathcal{B}$ should be thought of as an exhaustive listing of all the choice experiments that can be achieved, but it is possible that some subsets of $X$ are not achievable.}.
        \item[-] $C(\cdot)$ is a \textit{choice rule correspondence} that assigns a nonempty set of chosen elements $C(B)\subset B$, $\forall B\in \mathcal{B}$\footnote{The choice set $C(B)$ can contain a single element, which is the choice among the alternatives in $B$. BUT, $C(B)$ can contain multiple elements, then elements of $C(B)$ are the \textit{acceptable alternatives} in $B$.}.
    \end{enumerate}
\end{definition}

Now we discuss the CORE assumption in this section: the Weak Axiom of Revealed Preference (WARP):
\begin{definition}
    A choice set $(\mathcal{B},C(\cdot))$ satisfies WARP if:
    \begin{enumerate}
        \item[-] $\forall B,B^{\prime}$ and $x,y\in B\cap B^{\prime}$, $x\in C(B),y\in C(B^{\prime})\Rightarrow x\in C(B^{\prime})$
    \end{enumerate}
\end{definition}

Or in words, WARP requires that if $x$ is chosen from some alternatives where $y$ is also available, then there can be NO budget set containing both $x$ and $y$
but only $y$ is chosen.

Following WARP, define the \textit{reveal preference relation} $\succsim^*$ as:
\begin{definition}
    Given a choice structure $(\mathcal{B},C(\cdot))$, $x\succsim^*y\Leftrightarrow \exists B\in\mathcal{B}$ s.t. $x,y\in B\land x\in C(B)$
\end{definition}
In words, $x$ is revealed at least as good as $y$.

With revealed preference defiend, we can rephrase WARP as: \textit{If x is revealed at least as good as y, then y \textbf{cannot} be revealed preferred to x}. Hence, $\succsim^*$ is not symmetric.

One thing to remember is that $\succsim^*$ need not be either complete or transitive. For $\succsim^*$ to be comparable, for a $B\in\mathcal{B}$ and $x,y\in B$, we must have either $x\in C(B)$, $y\in C(B)$ or both.

\section{Linking Preferences with Choices}\label{chap2:sec3}
%Now we have two major approaches of decision making process: preference relations in Section \ref{chap1:sec1} and choice rules in Section \ref{chap1:sec2},
what we need to do is to link them. This linkage will emerge when we examine two central assumptions: \textbf{rationality} and \textbf{WARP}. So the major question here is: 
$$\textbf{rational}\succsim\xLeftrightarrow{???} (\mathcal{B},C(\cdot))\textbf{ satisfies WARP}$$
And the answer is: \textit{YES!} but not exactly. Now let's dig in.

\subsection*{Rational $\succsim\Rightarrow(\mathcal{B},C(\cdot))$ satisfies WARP}

First, $\textbf{rational}\succsim \Rightarrow (\mathcal{B},C(\cdot))\textbf{ satisfies WARP}$ is a big YES. To prove this, we need to define \textit{induced choice correspondence}:
\begin{definition}\label{def_induced_choice}
    Given a \textbf{rational} $\succsim$ on $X$, if the decision maker faces a nonempty subset of alternatives $B\subset X$, by maximizing her preference, she would choose any one of the elements in the 
    \textit{induced choice correspondence}: $C^*(B,\succsim)=\{x\in B:x\succsim y, \forall y\in B\}$
\end{definition}

The induced choice correspondence $C^*(B,\succsim)$ has an important property: 
\begin{theorem}
    if $X$ is finite, $C^*(B,\succsim)$ will be \textbf{nonempty}.
\end{theorem}

A brief proof of this proposition is: If $X$ is finite, $B$ is finite as well. We will prove by induction. Starting from $|B|=1$, the only element of $B$ is in $C^*(B,\succsim)$. Now suppose $C^*(B,\succsim)$ is nonempty when $|B_n|=n$, 
let $x^*\in C^*(B_n,\succsim)$; when $|B_{n+1}|=n+1$, let the $n+1$th element $y$ $(\{y\}=B_{n+1}\setminus B_n)$. By the completeness of a rational $\succsim$, either $y\succsim x^*$ or $x^*\succsim y$:
\begin{enumerate}
    \item[i.] $y\succsim x^*$: since $x^*\in C^*(B_n,\succsim)\Rightarrow x^*\succsim x, \forall x\in B_n$. By transitivity of $\succsim$, $y\succsim x,\forall\in B_n$. By completeness, $y\succsim y$ as well. Hence, $y\in C^*(B_{n+1},\succsim)$.
    \item[ii.] $x^*\succsim y$: since $x^*\in C^*(B_n,\succsim)\Rightarrow x^*\succsim x, \forall x\in B_n$, hence $x^*\succsim x, \forall x\in B_n\cup{y}\Rightarrow x^*\in C^*(B_{n+1},\succsim)$
\end{enumerate}

Notice that when $B$ is finite, a stronger condition of $\succsim$ being acyclic and complete is equilavent to an induced choice rule $C^*(B,\succsim)\neq \varnothing$: 
\begin{theorem}\label{theo_acyclic_choice}
    For a finite $B$, $\succsim$ is complete and \textbf{acyclic} $\Leftrightarrow C^*(B,\succsim)\neq \varnothing$
\end{theorem}
$\succsim$ is acyclic mean that: $b_1\succsim b_2,b_2\succsim b_3,\cdots, b_{n-1}\succsim b_n\Rightarrow b_n\not\succsim b_1$. An example of transitive but not \textit{acyclic} relations is indifference $\sim$: $a_1\sim a_2\sim \cdots\sim a_n\Rightarrow a_n\sim a_1$.
A brief proof of Theorem \ref{theo_acyclic_choice} is:
\begin{enumerate}
    \item[i.] acyclic $\succsim\Rightarrow C^*(B,\succsim)\neq\varnothing$: Suppose if $C^*(B,\succsim)=\varnothing$, for $b_1\in B$, $b_1\notin C^*(B,\succsim)\Rightarrow \exists b_2$ s.t. $b_2\succsim b_1$. Continue this process, we can generate a sequence of $\cdots\succsim b_2\succsim b_1$, since $B$ is finite, this sequence must end at $b_n$. If $\succsim$ is acyclic, $b_1\not\succsim b_n$, this gives $b_n\succ b_1$, which would mean $b_n$ must be in $C^*(B,\succsim)$, contradicting.
    \item[ii.] $C^*(B,\succsim)\neq\varnothing\Rightarrow$ acyclic $\succsim$: Suppose $\succsim$ is not acyclic, then there exists $b_1\succsim b_2\succsim \cdots\succsim b_n\succsim b_1$, then for set $B=\{b_1,b_2,\cdots,b_n\}$, $\nexists b^*$ s.t. $b^*\succsim b_i \forall b_i\in B$, i.e., $C^*(B,\succsim)=\varnothing$.
\end{enumerate}

With induced choice correspondence $C^*(B,\succsim)$ defined and non-emptyness proved, we can then say:
\begin{theorem}\label{thm_rational_leadto_WARP}
    If $\succsim$ is a rational preference relation, then the choice structure generated by $\succsim$, $(\mathcal{B},C^*(\cdot,\succsim))$, satisfies WARP
\end{theorem}

We can prove this theorem quite easily: $\forall B,B'$ suppose we have $x,y\in B\cap B'$ and $x\in C^*(B,\succsim),y\in C^*(B',\succsim)$, then $x\succsim a, \forall a\in B$ and $y\succsim b,\forall b\in B'$. Naturally, we have $x\succsim y$ since $y\in B$. By rationality (transitivity) of $\succsim$, we have $x\succsim y\succsim b,\forall b\in B'$, which means $x\in C^*(B',\succsim)$. This is precisely the definition of WARP

\subsection*{$(\mathcal{B},C(\cdot))$ satisfies WARP $\Rightarrow$ Rational $\succsim$}
The proof of this direction is more subtle, and is NOT necessarily a yes. Again, we start from a auxilary definition:
\begin{definition}\label{def_rationalize_choice}
    For a choice structure $(\mathcal{B},C(\cdot))$, a rational preference relation $\succsim$ \textbf{rationalizes} $C(\cdot)$ relative to $\mathcal{B}$ if $C(B)=C^*(B,\succsim), \forall B\in\mathcal{B}$.
\end{definition}

In words, if for all budget sets $B\in\mathcal{B}$, the choices generated by a rational $\succsim$, is just the choice rule $C(\cdot)$, $C(\cdot)$ is rationalized by $\succsim$. This is, in a sense, constructing an explanation of decision making behavior with preferences.

We already proved that $C^*(B,\succsim)$ satisfies WARP, which means that if a rationalizing preference relation to exist, WARP must be satisfied. However, if WARP is satisfied, a rationalizing preference relation does \textbf{NOT} necessarily exist.\footnote{A simple example is: $X=\{x,y,z\},\mathcal{B}=\{\{x,y\},\{y,z\},\{x,z\}\}$. Since $\mathcal{B}$ contains 3 binary menus, the choice structure $C(\{x,y\})=\{x\},C(\{y,z\})=\{y\},C(\{x,z\})=\{z\}$ vacuously satisfy WARP. But, this choice structure cannot be rationalized since it contradicts transitivity.}
Intuitiviely, more budget sets $B\in\mathcal{B}$ would mean that, to satisfy WARP, choice behavior would be restricted more, and it is easier to be self-contradicting. Therefore, to pin down a rational preference relation to rationalize $C(\cdot)$ relative to $\mathcal{B}$, we need to put some \textbf{restrictions on $\mathcal{B}$}.

\begin{theorem}\label{theorem_rationalizing_exist}
    If $(\mathcal{B},C(\cdot))$ is a choice structure that:
    \begin{enumerate}
        \item[i.] WARP is satisfied
        \item[ii.] $\mathcal{B}$ includes \textbf{all} subsets of $X$ of \textbf{up to 3} elements 
    \end{enumerate}
    then there exists a rational preference relation $\succsim$ s.t. $C(B)=C^*(B,\succsim),\forall B\in\mathcal{B}$. And this rational $\succsim$ is the \textbf{only} preference relation that can rationalize $(\mathcal{B},C(\cdot))$.
\end{theorem}

Now let's prove it, by examing the natural candidate for a rationalizing preference relation: the \textbf{revealed preference relation $\succsim^*$}:
\begin{enumerate}
    \item[\textbf{Step 1}] Prove that $\succsim^*$ is rational
    \begin{enumerate}
        \item[-] Completeness: By (ii) of Def.\ref{def_rationalize_choice}, all binary subsets of $X$ are in $\mathcal{B}$. Hence, $\{x,y\}\in\mathcal{B}$. For this binary menu, $C(\{x,y\})$ must contain either $x$ or $y$, therefore, $x\succsim^*y$ or $y\succsim^* x$ or both. Completeness proved.
        \item[-] Transitivity: $\forall\{x,y,z\}\in \mathcal{B}$, $C(\{x,y,z\})\neq \varnothing$. Suppose $x\succsim^*y,y\succsim^*z$, which implies that $x\in C(\{x,y\}),y\in C(\{y,z\})$, we then have three cases for $C(\{x,y,z\})$:
        \begin{enumerate}
            \item[a.] $x\in C(\{x,y,z\})$, WARP gives that $x\in C(\{x,z\})\Rightarrow x\succsim^*z$
            \item[b.] $y\in C(\{x,y,z\})$, we have $x\in C(\{x,y\})$. WARP gives $x\in C(\{x,y,z\})$ $\Rightarrow x\succsim^* z$
            \item[c.] $z\in C(\{x,y,z\})$, we have $y\in C(\{y,z\})$. WARP gives $y\in C(\{x,y,z\})$, and $x\in C(\{x,y\})$, WARP gives $x\in C(\{x,y,z\})\Rightarrow x\succsim^*z$
        \end{enumerate} 
        Hence, $x\succsim^*y,y\succsim^*z\Rightarrow x\succsim^* z$
    \end{enumerate} 
    \item[\textbf{Step 2}] Prove that $\succsim^*$ rationalizes $C(\cdot)$ on $\mathcal{B}$
    
    Now, we need to show $\forall B\in\mathcal{B}, C(B)=C^*(B,\succsim^*)$. Logically, this means the revealed preference $\succsim^*$ inferred from $C(\cdot)$ actually generates $C(\cdot)$. Formally, we prove it in 2 steps:
    \begin{enumerate}
        \item[a.] Suppose $x\in C(B)$, which means that $\forall y\in B, x\succsim^* y$ (by Def.\ref{def_revealed_pref}), hence $x\in C^*(B,\succsim^*)$ (by Def.\ref{def_induced_choice}). This proves $C(B)\subseteq C^*(B,\succsim^*)$
        \item[b.] Suppose $x\in C^*(B,\succsim^*)$, which means that $\forall y\in B, x\succsim^* y$ (by Def.\ref{def_induced_choice}). Therefore, $\forall y\in B$, there must exist a set $B_y\in\mathcal{B}$ s.t.
         $x,y\in B_y\Rightarrow x\in C(B_y)$. Since $C(B)\neq \varnothing$, suppose $z\in C(B)$, since $x\in C(B_z)$, WARP implies that $x\in C(B)$. This proves $C^*(B,\succsim^*)\subseteq C(B)$
    \end{enumerate}
    Together, we have $C(B)=C^*(B,\succsim^*)$.
    \item[\textbf{Step 3}] Prove $\succsim^*$ is the unique choice
    
    Since $\mathcal{B}$ includes all two-element subsets of $X$, the choice behavior in $C(\cdot)$ completely determines the pairwise preference relations over $X$ of any rationalizing preference.
\end{enumerate}

Now, it is \textbf{proved}! Notice that the main assumption(restriction) here is \textbf{$\mathcal{B}$ includes all subsets of $X$ of up to 3 elements}, this gives completeness, which is fundamental.

\subsection*{Two things to keep in mind}
We have proved the twoway links of preferences and choices:
\begin{enumerate}
    \item[-] Rational $\succsim\Rightarrow (\mathcal{B},C^*(\cdot,\succsim))$ satisfies WARP (see Thm.\ref{thm_rational_leadto_WARP})
    \item[-] A WARP-satisfying, up-to-3-element $(\mathcal{B},C(\cdot))$ can be uniquely rationalized by a rational $\succsim$ (see Thm.\ref{theorem_rationalizing_exist})
\end{enumerate}
However, there are still something to keep in mind.

First, for a given choice structure $(\mathcal{B},C(\cdot))$, there my be \textbf{more than one} rationalizing preference relation $\succsim$ in general. Here is the simplest example:
For $X=\{x,y\},\mathcal{B}\{\{x\},\{y\}\}$ and the choice structure $C(\{x\})=\{x\}, C(\{y\}=\{y\})$. In this case, \textbf{ANY} relation preference relation of $X$ can rationalize $C(\cdot)$
This is related to both Def.\ref{def_rationalize_choice} and (ii) of Thm.\ref{theorem_rationalizing_exist}. Thm.\ref{theorem_rationalizing_exist} gives that if $\mathcal{B}$ contains \textbf{ALL binary} menus of $X$,
then there could be at most one rationalizing preference relation.

Second, the restriction for WARP$\Rightarrow$ rational $\succsim$, namely $\mathcal{B}$ containing all subsets of up to 3 elements, is too strong. For many economic problems, we will not consider all possible subsets, or limit ourselves to up-to-3-element ones. A strengthened version of WARP will be introduced later for that purpose.

Finally, up till now, we define a rationalizing preference as one: $C(B)=C^*(B,\succsim)$ (Def.\ref{def_rationalize_choice}). A common alternative would be to require only $C(B)\subset C^*(B,\succsim)$: if $C(B)$ is a \textbf{subset} of the most preferred choices generated by $\succsim$, i.e., $C^*(B,\succsim)$. This will allow indifferences
to be more than the situation of anything might be picked. And it is empirically intuitive in a sense that observed choices will never fully reveal decision makers' entire preferencing maximizing choice set. Naturally, $C(B)\subset C^*(B,\succsim)$ is weaker than $C(B)=C^*(B,\succsim)$. But $C(B)\subset C^*(B,\succsim)$ has an interesting property: the all-indifferent preference
will be able to rationalize \textit{any} choice behavior. Therefore, when $C(B)\subset C^*(B,\succsim)$ is used, you would always need to put some additional restrictions on the rationalizing preference relation for the specific economic context.

\section{Introducing Utility}\label{chap2:sec4}
%Now, with preferences and choices defined, and the linkage between the two established, we need to transfer these concepts into math for analytic studies.
This is exactly why utility functions are introduced: to assign a number and rank the elements in $X$ according to preferences.

\begin{definition}
    A function $u:X\rightarrow\mathbb{R}$ is a \textit{utility function representing}
\end{definition}

\section{Commentary}\label{chap2:sec5}
%In this section, I discuss some of common commentaries on the standard preference model presented above.

\subsection{Preference model as a descriptive model}
A common complaint about the standard utility maximization/preference ranking model is that no one in reality actually calculates a number as utility before making choices.
This comment has a lot of sense to it since we rarely care about utility, let alone doing some math, before grocery shopping. But this observation does NOT invalidate the 
usefulness of preference/utility model.

The standard model does NOT regulate agents to consciously maximize utility, instead, it assumes individuals act \textit{as if} they maximize utility. Mathematically, we have already
proven that if choice behavior satisfies finite nonemptiness and WARP, then something will be chosen, and agents' choice behavior is just \textit{as if} it were preference driven, or the choice
behavior can be linked to a preference. If the set of choices is countable, then the preference-driven choice can be indexed by numbers, hence, becomes a mathematical question.

Utility/preference/choice system is considered as a description of choice behavior. Long as people do make a choice, and that choice satisfies WARP, we can always find a numerical way to 
\textit{describe} the behavioral pattern.

\subsection{Empirical limits}


\vspace{0.5cm}
\noindent\rule{\textwidth}{0.4pt}

For the content of this chapter, my main reference is Chapter 1 of \citet{mas1995microeconomic}. Section 1, Chapter 2 of \citet{kreps1990acourse} covers similar content but starts from strict preference $\succ$, it is a very
good complement to \citet{mas1995microeconomic}. Chapter 1 of \citet{kreps2013microeconomic} explores choice and preferences on infinite sets. Lecture 1 and 2 of \citet{ariel2012lecture} give a well organized, lecture-structured summary of
these contents, it is a very good read.