After examining the preference-based consumer theory, we have concluded that if a continuously differentiable demand function $x(\mathbf{p},w)$ is generated by rational preferences, this demand function must have
certain properties (see Thm.\ref{thm:properties_walrasian_demand}) and have a symmetric and negative semidefinite substitution matrix $S(\mathbf{p},w)$. In this section, we would examine the reverse:
if a demand function $x(\mathbf{p},w)$ has these properties, can it be rationalized by some preferences?

\subsection{Recover $\succsim$ from $x(\mathbf{p},w)$}\label{chp2:sec4:ssec1}
The recovering $\succsim$ from $x(\mathbf{p},w)$ will be done in 2 steps:
\begin{enumerate}
    \item[-] \textbf{Step 1}: recover $e(\mathbf{p},u)$ from $x(\mathbf{p},w)$
    \item[-] \textbf{Step 2}: recover $\succsim$ from $e(\mathbf{p},u)$  
\end{enumerate}
Here, demand is assumed to be single-valued.

\subsubsection*{Recover $e(\mathbf{p},u)$ from $x(\mathbf{p},w)$}

\subsubsection*{Recover $\succsim$ from $e(\mathbf{p},u)$}
The second step is to recover $\succsim$ given a function $e(\mathbf{p},u)$ that has the assumed properties: continuous, strictly increasing in $u$, non-decreasing, homogeneous of degree 1, and concave in $\mathbf{p}$ (see Thm.\ref{thm:properties_expenditure_func}).
Since demand is single-valued, $e(\mathbf{p},u)$ is differentiable.