For the welfare analysis, the goal is to determine the welfare change for a change in price from $\mathbf{p}^0$ to $\mathbf{p}^1$.

The key object used for welfare analysis is \myhl[blue!45!black]{\textbf{money metrics}}, which is built on the expenditure function: $e(\bar{\mathbf{p}},v(\mathbf{p},w))$. This expenditure function is \textbf{strictly increasing} in $v(\mathbf{p},w)$, hence is also an indirect utility function (a monotonic transformation).

With the expenditure function, we can construct two measures of welfare change:
\begin{definition}{Two Welfare Change Measures}{welfare_change}
    Let $u^0 = v(\mathbf{p}^0,w)$, $u^1 = v(\mathbf{p}^1,w)$. Naturally, $e(\mathbf{p}^0,u^0)=e(\mathbf{p}^1,u^1)=w$, define
    \begin{align*}
        EV(\mathbf{p}^0,\mathbf{p}^1,w) &=e(\mathbf{p}^0,u^1)-e(\mathbf{p}^0,u^0)=e(\mathbf{p}^0,u^1)-w & \text{equivalent variation}\\
        CV(\mathbf{p}^0,\mathbf{p}^1,w) &=e(\mathbf{p}^1,u^1)-e(\mathbf{p}^1,u^0)=w-e(\mathbf{p}^1,u^0) & \text{compensating variation}
    \end{align*}
\end{definition}

How to understand these 2 measures?

\begin{itemize}
    \item[-] EV: ${\color{red!55!black}e(\mathbf{p}^0,u^1)-e(\mathbf{p}^0,u^0)} = e(\mathbf{p}^0,v(\mathbf{p}^1,w))-e(\mathbf{p}^0,v(\mathbf{p}^0,w)) = {\color{red!55!black}e(\mathbf{p}^0,u^1)-w}$
    
    \underline{\textbf{interpretation}}: under the \myhl[red!55!black]{\textbf{old}} price, how much more wealth needed to get the new utility under the new price. A better way to write it is probably
    $$
    v \left(\mathbf{p}^0, w+EV \right)=u^1
    $$
    that is, EV is how much more wealth one would need to achieve the post-price-change utility level with the pre-price-change price level.
    
    \item[-] CV: ${\color{red!55!black}e(\mathbf{p}^1,u^1)-e(\mathbf{p}^1,u^0)} = e(\mathbf{p}^1,v(\mathbf{p}^1,w))-e(\mathbf{p}^1,v(\mathbf{p}^0,w)) = {\color{red!55!black}w- e(\mathbf{p}^1,u^0)}$
    
    \underline{\textbf{interpretation}}: under the \myhl[red!55!black]{\textbf{new}} price, how much wealth to give up to go back to the old utility under the old price. Similarly
    $$
    v \left(\mathbf{p}^1, w-CV \right)=u^0
    $$
    that is, CV is how much wealth one could save to achieve the post-price-change utility level with the pre-price-change price level.
\end{itemize}

Graphically,