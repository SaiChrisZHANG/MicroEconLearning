\subsection{Properties of Preferences Required}
To analyze demand with preferences (and later, utility), we first need some assumptions on preferences $\succsim$:

\paragraph*{$\succsim$ is rational}
\begin{enumerate}
    \item[-] \textbf{completeness}: $\forall \mathbf{x},\mathbf{y}\in X$, it must be either $\mathbf{x}\succsim \mathbf{y}$ or $\mathbf{y}\succsim \mathbf{x}$ or both
    \item[-] \textbf{transitivity}: $\forall \mathbf{x},\mathbf{y},\mathbf{z} \in X$ $\mathbf{x} \succsim \mathbf{y},\mathbf{y}\succsim \mathbf{z}\Rightarrow \mathbf{x}\succsim \mathbf{z}$ 
\end{enumerate}

\paragraph*{$\succsim$ is desirable}
It is reasonable to assume that larger amounts of commodities are preferred. So basically, the more the better.

Here, we have two common assumption, a stronger one, and a weaker one:
\begin{enumerate}
    \item[-] \textbf{monotone} (\textit{stronger}): if $\textbf{x},\textbf{y}\in X$, $\textbf{y}\gg \textbf{x}\Rightarrow \textbf{y}\succ \textbf{x}$
    
    \textbf{strongly monotone} (\textit{even stronger}): if $\textbf{x},\textbf{y}\in X$, $\textbf{y}\geq \textbf{x}$ and $\textbf{y}\neq \textbf{x}\Rightarrow \textbf{y}\succ \textbf{x}$
    \item[-] \textbf{locally nonsatiated} (\textit{weaker}): $\forall \mathbf{x} \in X, \forall \varepsilon >0$, $\exists \mathbf{y}\in X$ s.t. $\left\Vert \mathbf{y}-\mathbf{x} \right\Vert \leq \varepsilon$ and $\mathbf{y}\succ \mathbf{x}$
\end{enumerate}
Local nonsatiation is the common used one, since it poses minimal constraints. It is easy to see that \textbf{strongly monotone} $\Rightarrow$ \textbf{monotone} $\Rightarrow$ \textbf{locally nonsatiated}, here is a proof:
If $\succsim$ is strongly monotone and $\mathbf{x} \gg \mathbf{y}$, then $\mathbf{x}\geq \mathbf{y}$ and $\mathbf{x}\neq \mathbf{y}$ and $x\succ y$, thus $\succsim$ is monotone; 
if $\succsim$ is monotone, $\mathbf{x}\in X$ and $\varepsilon>0$, let $\mathbf{y}=\mathbf{x}+\frac{\varepsilon}{\sqrt{L}}\mathbf{e}$ where $\mathbf{e}=(1,\cdots,1)\in \mathbb{R}^L$, then $\left\Vert \mathbf{y}-\mathbf{x}\right\Vert\leq \varepsilon$ and $y\succ x$, hence $\succsim$ is locally nonsatiated.

$\succsim$ will divide $X$ into 3 sets relative to $\mathbf{x}$:
\begin{enumerate}
    \item[-] \textbf{upper contour set}: $\left\{\mathbf{y}\in X\mid \mathbf{y}\succsim \mathbf{x} \right\}$  
    \item[-] \textbf{lower contour set}: $\left\{\mathbf{y}\in X \mid \mathbf{x}\succsim \mathbf{y}\right\}$ 
    \item[-] \textbf{indifferent set}: $\left\{\mathbf{y}\in X \mid \mathbf{x}\sim \mathbf{y}\right\}$ 
\end{enumerate}
and \textbf{local nonsatiation} guarantees that the indifference set is a line.

\paragraph*{$\succsim$ is convex}
$\succsim$ is \textbf{convex} if $\forall \mathbf{x}\in X$, $\mathbf{y}\succsim\mathbf{x}$ and $\mathbf{z}\succsim\mathbf{x}$ $\Rightarrow$ $\alpha \mathbf{y}+(1-\alpha)\mathbf{z}\succsim \mathbf{x},\forall \alpha \in [0,1]$; that is, the upper contour set of $\mathbf{x}$, $\left\{\mathbf{y}\in X\mid \mathbf{y}\succsim \mathbf{x} \right\}$ is convex.

Convexity assumption is one of the central assumption, it expresses two intuitive observations of economic agents:
\begin{enumerate}
    \item[(i)] diminishing marginal rates of substitution
    \item[(ii)] inclination for diversification  
\end{enumerate}

$\succsim$ is \textbf{strictly convex} if $\forall \mathbf{x}\in X$, $\mathbf{y}\succsim\mathbf{x}$ and $\mathbf{z}\succsim\mathbf{x}$ $\Rightarrow$ $\alpha \mathbf{y}+(1-\alpha)\mathbf{z}\succ \mathbf{x},\forall \alpha \in [0,1]$.

\paragraph*{$\succsim$ is continuous}
So far, we have rationality, local nonsatiation and convexity, but these are NOT enough to guarantee a preference $\succsim$ to be representable by a utility funciton. An example is \textbf{lexicographic preference relation}.
%\begin{tcolorbox}[title = {lexicographic preference relation},colframe = black, colback = white]
%    This is my first \textbf{tcolorbox}.
%\end{tcolorbox}

The lexicographic preference relation is $\mathbf{x}\succsim \mathbf{y}\Leftrightarrow \left(x_1>y_1 \right) \lor \left( x_1=y_1 \land x_2> y_2\right) \lor \cdots \lor \left( x_1=y_1 \land \cdots x_{L-1}=y_{L-1} \land x_L\geq y_L \right)$. It is complete, transitive, strongly monotone and strictly convex; BUT, there is \textbf{NO} utility function
representing it\footnote{Here is a brief proof: Suppose there is a utility function $u(\cdot)$, then for every $x_1$, we can pick a rational number $r(x_1)$ s.t. $u(x_1,2)>r(x_1)>u(x_1,1)$, if $x_1 >x_1'$, then $r(x_1)>u(x_1,1)>u(x_1',2)>r(x_1')$. Hence $r(\cdot)$ is a one-to-one function from the set of real numbers (uncountably infinite) to the set of rational numbers (countably infinite), which is mathematically impossible. Another way of proving it is: suppose there is a utility function $u(\cdot)$, $[\inf u(x,\cdot),\sup u(x,\cdot)]$ is an interval, hence if $x<y$, $\sup u(x,\cdot)<\inf u(y,\cdot)$, which is impossible}.
To guarantee utility representability, $\succsim$ needs to be continuous: $\succsim$ is continuous if it is preserved under limits, that is, $\forall \left\{\{x^n,y^n\}\right\}^{\infty}_{n=1}$ with $x^n\succsim y^n, \forall n$, $x=\lim_{n\rightarrow\infty}x^n,y=\lim_{n\rightarrow\infty}y^n\Rightarrow x\succsim y$.

Continuity can also be expressed as: $\forall \mathbf{x}$, the upper contour set $\left\{\mathbf{y}\in X\mid \mathbf{y}\succsim \mathbf{x} \right\}$ and the lower contour set $\left\{\mathbf{y}\in X\mid \mathbf{x}\succsim \mathbf{y} \right\}$ are both \textbf{closed}. Here is the proof of this equivalence: 
\begin{enumerate}
    \item[-] continuity $\Rightarrow$ contour set closedness: $\forall \left\{y^n\right\}^{\infty}_{n=1}$, let $x^n=x,\forall n$, then if $x^n=x\succsim y^n$, by continuity, $y=\lim_{n\rightarrow \infty}y^n$, we have $x\succsim y$, hence the lower contour set is closed; the upper contour set can be proved the same way.
    \item[-] contour set closedness $\Rightarrow$ continuity: Suppose $\exists \left\{x^n\right\},\left\{y^n\right\}\in X$ s.t. $\forall n, x^n\succsim y^n$, $x^n\rightarrow x\in X, y^n\rightarrow y\in X$ and $y\succ x$. We know both $\left\{z\in X\mid y\succ z \right\}$ and $\left\{z\in X\mid z\succ x \right\}$ are open, hence $\exists N_1,N_2\in \mathbb{Z}^+$ s.t. $y\succ x^n,\forall n>N_1; y^n\succ x,\forall n>N_2$, conceivably, there are two cases on $\left\{y^n\right\}$:
    \begin{enumerate}
        \item[-] $\exists N_3 \in \mathbb{Z}^+$ s.t. $y^n\succsim y, \forall n>N_3$: then we have $y^n\succ x^n,\forall n>\max\{N_1,N_3\}$, contradiction.
        \item[-] $\exists \left\{y^{k(n)}\right\}$ s.t. $y\succ y^{k(n)},\forall n$: then $\exists m\in \mathbb{Z}^+$ s.t. $k(m)>N_2$, we know $\left\{z\in X\mid z\succ y^{k(m)}\right\}$ is open, $\exists N_4\in \mathbb{Z}^+$ s.t. $y^n\succ y^{k(m)}, \forall n>N_4$. Since $\left\{z\mid z\succsim y^{k(m)}\right\}$ is closed, $x\succsim y^{k(m)}$, however, since $k(m)>N_2$, $y^{k(m)}\succ x$ is assumed, hence contradiction.
    \end{enumerate} 
    Another way of proving this is to assume $\succsim$ to be monotone.
\end{enumerate}

\subsection{Utility representing preference}
Given the assumption of continuity, we can finally have the following thoerem:
\begin{theorem}
    If a preference relation $\succsim$ is rational and \textbf{continuous}, then there is a continuous utility function $u(x)$ that represents $\succsim$.
\end{theorem}

Here is the proof:
\begin{enumerate}
    \item[-] \underline{\textbf{continuous $u(\cdot)$ $\Rightarrow$ continuous preference}}: 
    
    suppose $\left\{x_n\right\}\rightarrow x^*,\left\{y_n\right\}\rightarrow y^*$, $u(\cdot)$ represents $\succsim$, hence $u(x_n)\geq u(y_n)$. If $u(\cdot)$ is continuous, $u(x_n)\rightarrow u(x^*),u(y_n)\rightarrow u(y^*)$, hence $u(x^*)\geq u(y^*)$, leading to $x^*\succsim y^*$.
    \item[-] \underline{\textbf{continuous preference $\Rightarrow$ continuous $u(\cdot)$}}: 
    
    The proof is done in 3 steps:
    \begin{enumerate}
        \item[Step 1] First, we construct the utility function.
        
        Since preferences are continuous, and if monotone is assumed, for each $\mathbf{x}\in \mathbb{R}^L$, define a function $\alpha(\mathbf{x}):\mathbb{R}^L\rightarrow \mathbb{R}$ such that $\alpha(\mathbf{x})=\inf\{a\mid a\cdot \mathbf{e}\succsim \mathbf{x}\}$ where $\mathbf{e}=(1,\cdots,1)^T$. By the continuity of $\succsim$, $\{a\mid a\cdot \mathbf{e}\succsim \mathbf{x}\}$ is non-empty and bounded below, hence $\inf$ exists.
        Also, continuity of $\succsim$ implies the upper contour set of $\mathbf{x}$ is closed, hence $\exists \underline{a}\in \mathbb{R}$ s.t. $\underline{a}=\inf\{a\mid a\cdot \mathbf{e}\succsim \mathbf{x}\}$. Let $\alpha(\mathbf{x})=\underline{a}$, $\alpha(\mathbf{x})\mathbf{e}\succsim \mathbf{x}$, also $\mathbf{x}\succ (\alpha(\mathbf{x})-\frac{1}{n})\mathbf{e}$, by continuity, $\mathbf{x}\succsim \alpha(\mathbf{x})$. Hence, $\mathbf{x}\sim \alpha(\mathbf{x})\mathbf{e}$\footnote{
            $\mathbf{x}\sim \alpha(\mathbf{x})\mathbf{e}$ can also be proven as: by continuity, $A^{up}=\{\alpha\in \mathbf{R}_+\mid \alpha\mathbf{e}\succsim \mathbf{x}\}$ and $A^{low}\{\alpha\in \mathbf{R}_+\mid \mathbf{x}\succsim\alpha\mathbf{e}\}$ are both non-empty and closed. By completeness of $\succsim$, $\mathbb{R}_+\subset A^{up}\cup A^{low}$; $\mathbb{R}_+$ is connected, hence $A^{up}\cap A^{low}\neq \varnothing$, thus there exists $\alpha$ s.t. $\alpha \mathbf{e}\sim \mathbf{x}$. And by monotonicity, $\alpha_1>\alpha_2\Rightarrow\alpha_1\mathbf{e}\succ \alpha_2\mathbf{e}$, hence the scalar $\alpha\mathbf{e}\sim\mathbf{x}$ is unique.
            }, then, we can take $\alpha(\mathbf{x})$ as the utility function $u(\mathbf{x})=\alpha(\mathbf{x})$.

        \item[Step 2] Next, we prove $\alpha(\mathbf{x})\geq \alpha(\mathbf{y})\Leftrightarrow \mathbf{x}\succsim\mathbf{y}$
            \begin{enumerate}
                \item[-] $\alpha(\mathbf{x})\geq \alpha(\mathbf{y})\Rightarrow \mathbf{x}\succsim\mathbf{y}$: Suppose $\alpha(\mathbf{x})\geq \alpha(\mathbf{y})$, by monotonicity, $\alpha(\mathbf{x})\mathbf{e}\succsim \alpha(\mathbf{y})\mathbf{e}$. Since $\mathbf{x}\sim \alpha(\mathbf{x})\mathbf{e},\mathbf{y}\sim\alpha(\mathbf{y})\mathbf{e}$, we have $\mathbf{x}\succsim \mathbf{y}$
                \item[-] $\alpha(\mathbf{x})\geq \alpha(\mathbf{y})\Leftarrow \mathbf{x}\succsim\mathbf{y}$: Suppose $\mathbf{x}\succsim \mathbf{y}$, then $\alpha(\mathbf{x})\mathbf{e}\sim\mathbf{x}\succsim \mathbf{y}\sim\alpha(\mathbf{y})\mathbf{e}$, by monotonicity, $\alpha(\mathbf{x})\geq \alpha(\mathbf{y})$
            \end{enumerate}
        
        \item[Step 3] Finally, we prove $\alpha(\mathbf{x})$ is continuous: $\lim_{n\rightarrow\infty}\mathbf{x}^n=\mathbf{x}\Rightarrow \lim_{n\rightarrow\infty}\alpha(\mathbf{x}^n)=\alpha(\mathbf{x}),\forall \{\mathbf{x}^n\}^{\infty}_{n=1}$
        \begin{enumerate}
            \item[-] $\{\alpha(\mathbf{x}^n)\}^{\infty}_{n=1}$ must have a convergent subsequence: by monotonicity, $\forall\epsilon>0$, $\forall \mathbf{x}'$ s.t. $\left\Vert\mathbf{x}'-\mathbf{x}\right\Vert\leq \epsilon$, $\alpha(\mathbf{x}')$ lies in a compact subset of $\mathbb{R}_+$, $[\alpha_0,\alpha_1]$. Since $\{\mathbf{x}^n\}^{\infty}_{n=1}$ converges to $\mathbf{x}$, then $\exists N$ s.t. $\forall n>N$, $\alpha(\mathbf{x}^n)$ lies in this compact set, hence, this infinite sequence must have a convergent subsequence.
            \item[-] all convergent subsequences of $\left\{\alpha(\mathbf{x}^n)\right\}^{\infty}_{n=1}$ converge to $\alpha(\mathbf{x})$: suppose otherwise, then there is a strictly increasing function $m(\cdot)$ that assigns to each $n$ a positive integer $m(n)$, and the subsequence $\left\{\alpha(\mathbf{x}^{m(n)}) \right\}^{\infty}_{n=1}$ converges to $\alpha'\neq \alpha(\mathbf{x})$. Without losing generality, let $\alpha'>\alpha(\mathbf{x})$, then by monotonicity, $\alpha'\mathbf{e}\succ \alpha(\mathbf{x})\mathbf{e}$. 
            Let $\hat{\alpha}=\frac{1}{2}[\alpha'+\alpha(\mathbf{x})]$, then $\hat{\alpha}(\mathbf{e})\succ \alpha(\mathbf{x})\mathbf{e}$. Since $\alpha(\mathbf{x}^{m(n)})\rightarrow\alpha'>\hat{\alpha}$, then $\exists \bar{N}$ s.t. $\forall n>\bar{N}$, $\alpha(\mathbf{x}^{m(n)})>\hat{\alpha}$, thus $\mathbf{x}^{m(n)}\sim \alpha(\mathbf{x}^{m(n)})\mathbf{e}\succ \hat{\alpha}\mathbf{e}$. By continuity of $\succsim$, we get $\mathbf{x}\sim \alpha(\mathbf{x})\mathbf{e}\succsim \hat{\alpha}\mathbf{e}$, leading to a contradiction. 
            With the same logic, $\alpha'<\alpha(\mathbf{x})$ is ruled out as well. Hence, all convergent subsequences $\left\{\alpha(\mathbf{x}^n)\right\}^{\infty}_{n=1}$ converge to $\alpha(\mathbf{x})$. Hence, $\lim_{n\rightarrow\infty}\alpha(\mathbf{x}^n)=\alpha(\mathbf{x})$.
        \end{enumerate}
    \end{enumerate}
\end{enumerate}

Now a continuous utility function can represent a continuous preference relation, there are several results on how they relate to each other:
\begin{enumerate}
    \item[-] continuous $u(\cdot)$ can represent $\succsim$, but any strictly increasing yet discontinuous transformation of $u(\cdot)$ may also represents $\succsim$
    \item[-] preference $\succsim$ is monotone $\Rightarrow$ utiity function $u(\cdot)$ is increasing: $\mathbf{x}\gg \mathbf{y}\Rightarrow u(\mathbf{x})>u(\mathbf{y})$ 
    \item[-] preference $\succsim$ is convex $\Rightarrow$ utility function $u(\cdot)$ is quasiconcave: the upper contour set of $\mathbf{x}$, 
    $\left\{\mathbf{y}\in\mathbb{R}^L_+\mid \mathbf{y}\succsim \mathbf{x} \right\}$ is convex $\Rightarrow$ $u(\alpha\mathbf{x}+(1-\alpha)\mathbf{y})\geq \min\{u(\mathbf{x}),u(\mathbf{y})\}$. 
    It is \textbf{quasiconcave}, NOT concave.
\end{enumerate}
Increasingness and quasiconcavity are both ordinal properties of $u(\cdot)$, preserved for any increasing transformation.

\subsection{UMP (Utility Maximizing Problem)}
A consumer's problem is to choose a feasible consumption bundle given a positive price-wealth combination $(\mathbf{p},w)$ to maximize her utility:
$$\max_{\mathbf{x}\geq 0} u(\mathbf{x}),\ \text{s.t.}\ \mathbf{p}\cdot\mathbf{x}\leq w$$
if $\mathbf{p}\gg 0$, $u(\cdot)$ is continuous, then this problem has a solution: the Walrasian budget set $B_{\mathbf{p},w}=\left\{\mathbf{x}\in\mathbb{R}^L_+\mid \mathbf{p}\cdot\mathbf{x}\leq w\right\}$ is a compact set, hence a continuous function on it always has a maximum value.

This problem will induce two objects: Walrasian demand correspondence/function $\mathbf{x}(\mathbf{p},w)\in \mathbb{R}^L_+$ and the value function $V(\mathbf{p},w)$.

\subsubsection*{Walrasian demand $\mathbf{x}(\mathbf{p},w)$}
Suppose continuous $u(\cdot)$, representing locally nonsatiated $\succsim$, $\mathbf{x}(\mathbf{p},w)$ has the following properties:
\begin{enumerate}
    \item \textit{\textbf{homogeneity of degree zero}} in $(\mathbf{p},w)$: $\mathbf{x}(\alpha\mathbf{p},\alpha w)=\mathbf{x}(\mathbf{p},w),\forall \mathbf{p}\gg 0,w>0,\alpha>0$
    \item \textit{\textbf{Walras' law}}: $\mathbf{p}\cdot \mathbf{x}(\mathbf{p},w)=w$. This is guaranteed by the local nonsatiation assumption.
    \item \textit{\textbf{convexity/uniqueness}}:
    \begin{enumerate}
        \item[(a)] if $\succsim$ is convex, then $u(\cdot)$ is quasiconcave, hence $\mathbf{x}(\mathbf{p},w)$ is a convex set
        
        \textit{Proof}: suppose two maximizers $\mathbf{x}_1,\mathbf{x}_2\in\mathbf{x}(\mathbf{p},w)$, since Walrasian budget set is convex, hence $\alpha \mathbf{x}_1+(1-\alpha)\mathbf{x}_2$ is feasible. $u(\mathbf{x}_1)=u(\mathbf{x}_2)=u_{\max}$, hence by quasiconcavity of $u(\cdot)$, $u(\alpha \mathbf{x}_1+(1-\alpha)\mathbf{x}_2)\geq u_{\max}$, hence $\alpha \mathbf{x}_1+(1-\alpha)\mathbf{x}_2\in \mathbf{x}(\mathbf{p},w)$.
        \item[(b)] if $\succsim$ is strictly convex, then $u(\cdot)$ is strictly quasiconcave, hence $\mathbf{x}(\mathbf{p},w)$ is single-valued
        
        \textit{Proof}: suppose two maximizers $\mathbf{x}_1,\mathbf{x}_2\in\mathbf{x}(\mathbf{p},w)$, again $\alpha \mathbf{x}_1+(1-\alpha)\mathbf{x}_2$ is feasible. $u(\mathbf{x}_1)=u(\mathbf{x}_2)=u_{\max}$, hence by stirct quasiconcavity of $u(\cdot)$, $u(\alpha \mathbf{x}_1+(1-\alpha)\mathbf{x}_2)> u_{\max}$, hence $\mathbf{x}_1,\mathbf{x}_2$ are not maximizers, i.e., there are at most one maximizer (and there must be a maximizer in the compact Walrasian set).
    \end{enumerate}
    \item \textit{\textbf{continuity}}: Walrasian demand $\mathbf{x}(\mathbf{p},w)$ is the solution of the utility maximizing problem, then by maximum theorem, if the utility function is continuous (preference is continuous), its maximizer $\mathbf{x}(\mathbf{p},w)$ is also continuous. 
    
    This property can also be proved in a more "economic" way: If $\mathbf{x}(\mathbf{p},w)$ is not continuous, then $\exists\left\{p^n\right\}\rightarrow p^*$ such that $\mathbf{x}(\mathbf{p}^*,w)=\mathbf{x}^*$, but $\mathbf{x}(p^n,w)\not\rightarrow \mathbf{x}^*$, or in an $\epsilon$ representation: $\exists \epsilon>0$ s.t. $\left\Vert \mathbf{x}(\mathbf{p}^n,w),\mathbf{x}^* \right\Vert >\epsilon$. 
    We know $\mathbf{x}(\mathbf{p}^n,w)$ is in a compact set ($p^n$ is bounded away from 0 and by wealth $w$), hence we can assume, without loss of generality, $\exists \mathbf{y}^*\neq \mathbf{x}^*, \mathbf{x}(\mathbf{p}^n,w)\rightarrow \mathbf{y}^*$. Since $\forall n,\mathbf{p}^n\mathbf{x}(\mathbf{p}^n,w)\leq w$, $\mathbf{p}^*\mathbf{y}^* \leq w$, at the same time, $\mathbf{x}^*$ is the utility maximizer given the Walrasian budget, hence $\mathbf{x}^*\succ \mathbf{y}^*$.
    By the continuity of preferences, this $\succ$ is preserved in the small neighborhoods of $\mathbf{x}^*$ and $\mathbf{y}^*$: for sufficiently large $n$, $\mathbf{x}(\mathbf{p}^n,w)$ is in the neighborhodd of $y^*$; we can choose a bundle $\mathbf{z}^*$ in the neighorhood of $\mathbf{x}^*$ s.t. $\mathbf{p}^* \mathbf{z}^*<w$. For sufficiently large $n$, we have $\mathbf{p}^n\mathbf{z}^*<w$; and at the same time $\mathbf{z}^*\succ \mathbf{x}(\mathbf{p}^n,w)$. These two: $\mathbf{p}^n\mathbf{z}^*<w,\mathbf{z}^*\succ \mathbf{x}(\mathbf{p}^n,w)$ are directly contradicting each other (monotonicity).
\end{enumerate}

In general, with the Lagrangean $\mathcal{L}=u(\mathbf{x})+\lambda (w-\mathbf{p}\cdot\mathbf{x})$, there are two cases of optimum $\mathbf{x}(\mathbf{p},w)$:
\begin{enumerate}
    \item[-] interior optimum ($\forall x_l>0$): by solving the Lagrangean, get FOC: $\partial u(\mathbf{x}^*)/\partial x_l= \lambda p_l,\forall l\in \left\{1,\cdots,L\right\}$. Or, write in matrix notation: $\nabla u(\mathbf{x}^*)=\lambda \mathbf{p}$ where the gradient vector of $u(\cdot)$ at $\mathbf{x}$ is $\nabla u(\mathbf{x})=\left[\partial u(\mathbf{x})/\partial x_1,\cdots, \partial u(\mathbf{x})/\partial x_L\right]$.
    
    At the interior optimum, we have the marginal rate of substitution of $l$ for $k$ at $\mathbf{x}^*$ as $MRS_{lk}(\mathbf{x}^*)=\frac{\partial u(\mathbf{x}^*)/\partial x_l}{\partial u(\mathbf{x}^*)/\partial x_k}=\frac{p_l}{p_k}$, it is equal to the price ratio. Graphically, $\mathbf{x}^*$ is the tangent point of the utility curve to the budget line.
    \item[-] border optimum ($\exists x_l=0$): when there is no interior optimum, $\nabla u(\mathbf{x}^*)$ is not proportional to prices, particularly, $\partial u(\mathbf{x}^*)/\partial x_l\begin{cases}\leq 0, & x^*_l=0 \\ =0 , & x^*_l>0\end{cases}$. At the border, due to nonnegativity of consumption, even when the agent wants to decrease the consumption of $x_l$, she won't be able to.
\end{enumerate}

The Lagrange multiplier $\lambda$ itself is very interesting as well. $\lambda$ gives the marginal value of relaxing the constraint in the UMP, hence it is the consumer's \textbf{marginal utility value of wealth} at the optimum: $$\nabla u(\textbf{x}(\textbf{p},w))\cdot \mathrm{D}_w\mathbf{x}(\mathbf{p},w)=\underbrace{\lambda \mathbf{p}\cdot \mathrm{D}_w\mathbf{x}(\mathbf{p},w)=\lambda}_{\text{By Walras' law: }\mathbf{p}\cdot\mathbf{x}(\mathbf{p},w)=w}$$ where $\mathrm{D}_w\mathbf{x}(\mathbf{p},w)=\left[\partial x_1(\mathbf{p},w)/\partial w,\cdots, \partial x_L(\mathbf{p},w)/\partial w\right]$

But here we assume the Lagrangean multiplier is fixed across all commodities, which is a very strong assumption. Hence, we could perhaps think of another way to select $\lambda$: 
Think Lagrangean multiplier as a penalty mechanism such that any deviation from the constraint will decrease utility, hence, $\lambda$ can only be chosen from $\left\{\frac{\partial u(\mathbf{x})}{\partial x_l}\cdot\frac{1}{p_l} \right\}^L_{l=1}$; at the same time, $\lambda$ reflects the marginal utility of wealth, 
so it is reasonable to choose a value that is as large as possible. Therefore, we can choose $\lambda = \max_l \frac{\partial u(\mathbf{x})}{\partial x_l}\cdot \frac{1}{p_l}$. Choosing such $\lambda$, we have the Lagrangean as: $$ \mathcal{L} = u(\mathbf{x}) + \frac{\partial u(\mathbf{x})}{\partial x_{l^*}}\cdot\frac{1}{p_{l^*}}\left(w-\mathbf{p}\cdot\mathbf{x} \right) = \underbrace{\frac{\partial u(\mathbf{x})}{\partial x_{l^*}}\cdot\frac{1}{p_{l^*}}w}_{\equiv U^*} + u(\mathbf{x}) - \frac{\partial u(\mathbf{x})}{\partial x_{l^*}}\cdot\frac{1}{p_{l^*}}\cdot \mathbf{p}\cdot\mathbf{x}$$
If $u(\mathbf{x})$ is homogeneous of degree $m$, then by Euler's theorem\footnote{Euler's theorem: For a function $f(\mathbf{x})$ homogeneous of degree $m$ (i.e. $f(t\mathbf{x})=t^mf(\mathbf{x}),\forall t\neq 0$), if $f$ has all partial derivatives of first order, then $\sum^L_{l=1}x_l\frac{\partial f}{\partial x_l}=mf(\mathbf{x})$. A brief proof is: define $x'_l=x_l t,\forall l$, then $t^mf(\mathbf{x})=f(t\mathbf{x})\xRightarrow{\partial t} mt^{m-1}f(\mathbf{x})=\sum^L_{l=1}\frac{\partial f}{\partial x'_l}\frac{\partial x'_l}{\partial t}=\sum^L_{l=1}\frac{\partial f}{\partial x'_l}x_l$.
If we choose $t=1$, thus $x'_l=x_l$, we would have Euler's theorem.}, we have 
    $$\mathcal{L} = U^* +u(\mathbf{x})- \frac{\partial u(\mathbf{x})}{\partial x_{l^*}}\cdot\frac{1}{p_{l^*}}\cdot \left(\sum_{l=1}^L p_lx_l \right) \leq U^* + u(\mathbf{x})- \left(\sum_{l=1}^L \frac{\partial u(\mathbf{x})}{\partial x_{l}} x_l\right) = U^* +(1-m)u(\mathbf{x})$$
this will be an upper bound for utility, it is achieved when the agent spends everything on commodity $l^*$, i.e., purchase $\frac{w}{p_{l^*}}$. This is feasible, and achieves the upper bound.

In general, solving for $\mathbf{x}(\mathbf{p},w)$ is a constrained optimization problem, I will discuss it more thoroughly in the later chapter.

\subsubsection*{Indirect utility function $v(\mathbf{p},w)$}
The indirect utility function $v(\mathbf{p},w)\in \mathbb{R}$ is the maximized utility, i.e., 
$$v(\mathbf{p},w)=u(\mathbf{x}^*),\forall \mathbf{x}^*\in \mathbf{x}(\mathbf{p},w)$$
it has the following properties, closely related to those of Walrasian demand $\mathbf{x}(\mathbf{p},w)$:

\begin{enumerate}
    \item \textit{\textbf{homogeneity of degree zero}} in $(\mathbf{p},w)$: $\mathbf{x}(\alpha\mathbf{p},\alpha w)=\mathbf{x}(\mathbf{p},w),\forall \mathbf{p}\gg 0,w>0,\alpha>0$
    \item \textit{\textbf{strictly increasing in $w$, nonincreasing in $p_l,\forall l$}}: 
    \item \textit{\textbf{quasiconvexity}}: $\forall \bar{v}$, the set $\left\{(\mathbf{p},w):v(\mathbf{p},w)\leq \bar{v}\right\}$ is convex
    \item \textit{\textbf{continuity}}: $v(\mathbf{p},w)$ is continuous in $\mathbf{p}$ and $w$.
\end{enumerate}

\subsection{EMP (Expenditure Minimizing Problem)}