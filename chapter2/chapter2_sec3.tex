\subsection{Properties of Preferences Required}
To analyze demand with preferences (and later, utility), we first need some assumptions on preferences $\succsim$:

\paragraph*{$\succsim$ is rational}
\begin{enumerate}
    \item[-] \textbf{completeness}: $\forall \mathbf{x},\mathbf{y}\in X$, it must be either $\mathbf{x}\succsim \mathbf{y}$ or $\mathbf{y}\succsim \mathbf{x}$ or both
    \item[-] \textbf{transitivity}: $\forall \mathbf{x},\mathbf{y},\mathbf{z} \in X$ $\mathbf{x} \succsim \mathbf{y},\mathbf{y}\succsim \mathbf{z}\Rightarrow \mathbf{x}\succsim \mathbf{z}$ 
\end{enumerate}

\paragraph*{$\succsim$ is desirable}
It is reasonable to assume that larger amounts of commodities are preferred. So basically, the more the better.

Here, we have two common assumption, a stronger one, and a weaker one:
\begin{enumerate}
    \item[-] \textbf{monotone} (\textit{stronger}): if $\textbf{x},\textbf{y}\in X$, $\textbf{y}\gg \textbf{x}\Rightarrow \textbf{y}\succ \textbf{x}$
    
    \textbf{strongly monotone} (\textit{even stronger}): if $\textbf{x},\textbf{y}\in X$, $\textbf{y}\geq \textbf{x}$ and $\textbf{y}\neq \textbf{x}\Rightarrow \textbf{y}\succ \textbf{x}$
    \item[-] \textbf{locally nonsatiated} (\textit{weaker}): $\forall \mathbf{x} \in X, \forall \varepsilon >0$, $\exists \mathbf{y}\in X$ s.t. $\left\Vert \mathbf{y}-\mathbf{x} \right\Vert \leq \varepsilon$ and $\mathbf{y}\succ \mathbf{x}$
\end{enumerate}
Local nonsatiation is the common used one, since it poses minimal constraints. It is easy to see that \textbf{strongly monotone} $\Rightarrow$ \textbf{monotone} $\Rightarrow$ \textbf{locally nonsatiated}, here is a proof:
If $\succsim$ is strongly monotone and $\mathbf{x} \gg \mathbf{y}$, then $\mathbf{x}\geq \mathbf{y}$ and $\mathbf{x}\neq \mathbf{y}$ and $x\succ y$, thus $\succsim$ is monotone; 
if $\succsim$ is monotone, $\mathbf{x}\in X$ and $\varepsilon>0$, let $\mathbf{y}=\mathbf{x}+\frac{\varepsilon}{\sqrt{L}}\mathbf{e}$ where $\mathbf{e}=(1,\cdots,1)\in \mathbb{R}^L$, then $\left\Vert \mathbf{y}-\mathbf{x}\right\Vert\leq \varepsilon$ and $y\succ x$, hence $\succsim$ is locally nonsatiated.

$\succsim$ will divide $X$ into 3 sets relative to $\mathbf{x}$:
\begin{enumerate}
    \item[-] \textbf{upper contour set}: $\left\{\mathbf{y}\in X\mid \mathbf{y}\succsim \mathbf{x} \right\}$  
    \item[-] \textbf{lower contour set}: $\left\{\mathbf{y}\in X \mid \mathbf{x}\succsim \mathbf{y}\right\}$ 
    \item[-] \textbf{indifferent set}: $\left\{\mathbf{y}\in X \mid \mathbf{x}\sim \mathbf{y}\right\}$ 
\end{enumerate}
and \textbf{local nonsatiation} guarantees that the indifference set is a line.

\paragraph*{$\succsim$ is convex}
$\succsim$ is \textbf{convex} if $\forall \mathbf{x}\in X$, $\mathbf{y}\succsim\mathbf{x}$ and $\mathbf{z}\succsim\mathbf{x}$ $\Rightarrow$ $\alpha \mathbf{y}+(1-\alpha)\mathbf{z}\succsim \mathbf{x},\forall \alpha \in [0,1]$; that is, the upper contour set of $\mathbf{x}$, $\left\{\mathbf{y}\in X\mid \mathbf{y}\succsim \mathbf{x} \right\}$ is convex.

Convexity assumption is one of the central assumption, it expresses two intuitive observations of economic agents:
\begin{enumerate}
    \item[(i)] diminishing marginal rates of substitution
    \item[(ii)] inclination for diversification  
\end{enumerate}

$\succsim$ is \textbf{strictly convex} if $\forall \mathbf{x}\in X$, $\mathbf{y}\succsim\mathbf{x}$ and $\mathbf{z}\succsim\mathbf{x}$ $\Rightarrow$ $\alpha \mathbf{y}+(1-\alpha)\mathbf{z}\succ \mathbf{x},\forall \alpha \in [0,1]$.

\paragraph*{$\succsim$ is continuous}
So far, we have rationality, local nonsatiation and convexity, but these are NOT enough to guarantee a preference $\succsim$ to be representable by a utility funciton. An example is \textbf{lexicographic preference relation}.
%\begin{tcolorbox}[title = {lexicographic preference relation},colframe = black, colback = white]
%    This is my first \textbf{tcolorbox}.
%\end{tcolorbox}

The lexicographic preference relation is defined as $\mathbf{x}\succsim \mathbf{y}$