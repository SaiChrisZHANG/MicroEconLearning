We have already defined the economic-affordability constraint of consumers, if we also limit consumption bundle $x$ to be non-negative, we would have the Walrasian, or competitive budget:

\begin{definition}
    The Walrasian, or competitive budget set $$B_{\vec{p},w}=\{x\in\mathbb{R}^L_+:\vec{p}\cdot\vec{x}\leq w\}$$ is the set of all feasible consumption bundles give market prices $\vec{p}$ and wealth $w$.
\end{definition}

From a Walrasian budget's point of view, a consumer can only choose a consumption bundle $x$ from $B_{\vec{p},w}$. An underlining assumption here is $w>0$, otherwise consumers cannot afford anything. We can also
separately define the "edge" of a Walrasian budget set as:
\begin{definition}
    The \textit{budget hyperplane} is the set $\{x\in\mathbb{R}^L_+:\vec{p}\cdot\vec{x}=w\}$
\end{definition}
It determines the upper bound of the budget set: with prices of other commodities ($\vec{p}_{-i}$) and wealth level $w$ fixed, the change of commodity $i$'s price $p_i$ will enlarge/shrink the budget set by moving the budget hyperplane.
Geometrically, the price vector $\vec{p}$ must be orthogonal to the budget hyperplane, we can think it this way: for any two bundles $\vec{x}$ and $\vec{x}'$ one the budget hyperplane, we must have $\vec{p}\cdot\vec{x}=\vec{p}\cdot\vec{x}'=w$, hence
$\vec{p}\cdot\Delta\vec{x}=\vec{p}\cdot(\vec{x}'-\vec{x})=0$ is always true.

A core feature of the Walrasian budget set is that it is \textbf{convex}: $\forall \vec{x},\vec{y}\in B_{\vec{p},w}, \alpha\in[0,1], \alpha\vec{x}+(1-\alpha)\vec{y}\in B_{\vec{p},w}$. This is very easy to prove: $\vec{x}\in\mathbb{R}^L_+\land\vec{y}\in\mathbb{R}^L_+\Rightarrow \alpha\vec{x}+(1-\alpha)\vec{y}\in \mathbb{R}^L_+$, $\vec{p}\cdot\vec{x}\leq w\land\vec{p}\cdot \vec{y}\leq w\Rightarrow \alpha(\vec{p}\cdot\vec{x})+(1-\alpha)(\vec{p}\cdot\vec{y})\leq w$.
Notice that the Walrasian budget set is not automatically convex. Its convexity is induced from the convexity of its superset (the consumption set), in this case $\mathbb{R}^L_+$. In general, it is easy to show that the Walrasian budget set will convex as long as its corresponding consumption set is convex.

Of course, it is perfectly possible that a consumer's budget is NOT convex (and not Walrasian, in that sense), the brilliant work of \citet[]{deaton1980economics} has documented and discussed many complicated consumption sets that are not convex.

