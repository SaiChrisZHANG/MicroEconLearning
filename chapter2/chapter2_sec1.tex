First, we introduce the basic settings of a consumer's problem in a market economy. These concepts will keep reoccuring in the following sections.

\subsection{Commodities}
First, we need to define the goods and services the consumers consume. We do not actually care about what they specifically are, instead, we use a very abstract concept \textit{commodities}
to summarize and analyze them.
\begin{definition}
    Assume there are $L<\infty$ different commodities, a \textit{commodity vector} or \textit{commodity bundle} is a list of amounts of the different commodities:$$x=\left[x_1,\cdots,x_L\right]^T$$
\end{definition}
$x$ can be view as a point in an $\mathbb{R}^L$ space, i.e., the commodity space. Each entry $x_l$ of $x$ ($l=1,\cdots,L$) represents the amount of commodity $l$ consumed, hence, the vector is referred to
as \textit{consumption vector} or \textit{consumption bundle}.

Three things to keep in mind:
\begin{enumerate}
    \item[-] Time can be incorporated into this setting, namely, today's commodity is distinct from tomorrow's commodity, even if they are otherwise the same. The value of time will come back in later chapters and is crucial in a large strand of behavioral economic literature. Same logic applies to other limitations that are easily neglected, like geographic ones.
    \item[-] Negative entries can exists in a commodity vector, indicating debits or net outflows of goods. In a producing problem or exchange problem, negative entries of commodity vectors are not rare.
    \item[-] Consumption is quite flexible and comes in many format empirically, for the sake of data collection conveniency, consumption data are often aggregated monthly, quarterly for even annually. Meanwhle, some consumptions in the commodity vectors may not actually occur in the market.
\end{enumerate}

\subsection{The Consumption Set}
Consumptions are limited by a number of constraints, which will form a subset of commodity space $X\subset \mathcal{R}^L$. With in this subset, all possible commodity bundles can be 
consumed, this is exactly the definition of consumption sets (see \citet[Page 19-20]{mas1995microeconomic} for some simple examples of consumption sets).

For now, we will focus on the simplest consumption set: all possible non-negative commodity bundles:
$$X=\mathbb{R}^L_+=\{x\in \mathbb{R}^L: x_l\geq 0,\forall l=1,\cdots, L\}$$

It is easy to show that
$$\mathbb{R}^L_+ \text{ is a }\textbf{convex} \text{ set}$$

A brief proof: $\forall \vec{x},\vec{y}\in \mathbb{R}^L_+$ and $\forall\alpha \in [0,1]$, $\alpha\vec{x}+(1-\alpha)\vec{y}=[\alpha x_1+(1-\alpha)y_1,\cdots,$ $\alpha x_L+(1-\alpha)y_L]^T$. Since $x_i\geq 0,y_i\geq 0$, $\alpha x_i+(1-\alpha)y_i\geq 0 \Rightarrow\alpha\vec{x}+(1-\alpha)\vec{y}\in \mathbb{R}^L_+$.

Convexity of consumption sets is an essential assumption here, but some of the results do survive without the assumption fo convexity.

Although consuption sets are formed due to some constraints, but these constraints have nothing to do with consumers' budget (exogenous constraints). It is intuitive that with a large enough budget (infinitely large if you may), you can always afford any consumption bundle in a give consumption set. But what if consumers, as in reality, do have a budget constraint and cannot afford every bundle in the consumption set?

\subsection{Prices and Consumption Cost}
Budget constraints are an important economic constraint faced by consumers: one can only consume the commodity bundles that she can afford.

To formalize this constraint, we need to introduce the \textit{price vector}: 
$$\vec{p}=\left[p_1,\cdots,p_L\right]\in \mathbb{R}^L$$
This price vector contains unit price information for each of the $L$ commodities. They are all traded in the market and the price information of them is publicly quoted (the \textit{principle of completeness of markets}).
For simplicity, we assume $\vec{p}\gg 0$ i.e. $\forall l,p_l>0$\footnote{Of course, price can be negative, meaning that consumers are actually paid to consume the "bad "commodity, such as polution.}.

Another important assumption is the \textit{price-taking assumption}: consumers do NOT have the power to influence the prices. Here, each consumer only buys a small (neglectable) fraction of the total demand for commodities.

With prices defined, we can finally define the \textbf{economic-affordability constraint} of consumers: For a consumer with wealth $w$, a consumption bundle $\vec{x}\in \mathbb{R}^L_+$ is affordable if its total cost does NOT exceed the consumer's wealth level $w$, formally, $$\vec{p}\cdot \vec{x}=p_1x_1+p_2x_2+\cdots+p_Lx_L\leq w$$
With the two core assumptions stated above, consumers face a \textit{linear price schedule}.